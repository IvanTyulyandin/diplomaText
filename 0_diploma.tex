% Тут используется класс, установленный на сервере Papeeria. На случай, если
% текст понадобится редактировать где-то в другом месте, рядом лежит файл matmex-diploma-custom.cls
% который в момент своего создания был идентичен классу, установленному на сервере.
% Для того, чтобы им воспользоваться, замените matmex-diploma на matmex-diploma-custom
% Если вы работаете исключительно в Papeeria то мы настоятельно рекомендуем пользоваться
% классом matmex-diploma, поскольку он будет автоматически обновляться по м⎄ере внесения корректив
%

% По умолчанию используется шрифт 14 размера. Если нужен 12-й шрифт, уберите опцию [14pt]
%\documentclass[14pt]{matmex-diploma}
\documentclass[14pt]{matmex-diploma-custom}

\hyphenation{op-tical net-works semi-conduc-tor con-tract Et-he-re-um}
\hyphenation{пре-по-да-ва-тель сту-дент вы-чис-ли-тель-но}
\newcommand\name[1]{\textsc{#1}}
\newcommand\achtung[1]{{\color{red}#1}}

\usepackage{amssymb,amsmath,cancel,cite,color,cmap,float,
	graphicx, multirow,pgfplots,tikz,wrapfig,xcolor,caption, 
	subcaption}

% tikz settings
\usetikzlibrary{shapes,arrows,positioning}
\tikzset{
	node distance=5cm,
	every state/.style={ 
		semithick,
		fill=gray!10
	},
	every edge/.style={
		draw,
		->,>=stealth, 
		auto,
		semithick
	}
}
\pgfplotsset{compat=1.3}

\usepackage{graphicx}
\usepackage{hyperref}
\hypersetup{
    colorlinks=true,
    linkcolor=black,
    filecolor=magenta,
    urlcolor=blue,
}
% code listings settings
\usepackage{listings}
% replace Listing in code caption
\renewcommand{\lstlistingname}{Листинг}
\usepackage{xcolor}

\definecolor{codegreen}{rgb}{0,0.6,0}
\definecolor{codegray}{rgb}{0.5,0.5,0.5}
\definecolor{codepurple}{rgb}{0.58,0,0.82}
\definecolor{backcolour}{rgb}{0.95,0.95,0.92}

\lstdefinestyle{mystyle}{
    backgroundcolor=\color{backcolour},   
    commentstyle=\color{codegreen},
    keywordstyle=\color{magenta},
    numberstyle=\tiny\color{codegray},
    stringstyle=\color{codepurple},
    basicstyle=\ttfamily\footnotesize,
    breakatwhitespace=false,         
    breaklines=true,                 
    captionpos=b,                    
    keepspaces=true,                 
    numbers=left,                    
    numbersep=5pt,                  
    showspaces=false,                
    showstringspaces=false,
    showtabs=false,                  
    tabsize=2
}

\lstdefinelanguage{my_pseudo} {
	morekeywords={function, for, return, let, in},
	sensitive=false,
	morecomment=[l]{//},
	morecomment=[s]{/*}{*/},
	morestring=[b]",
}

\lstset{style=mystyle}
\lstset{language=my_pseudo}

\begin{document}
% Год, город, название университета и факультета предопределены,
% но можно и поменять.
% Если англоязычная титульная страница не нужна, то ее можно просто удалить.
\filltitle{ru}{
    chair              = {Программная инженерия\\ Кафедра системного программирования},
    title              = {Исследование применимости специализации алгоритма Витерби скрытой марковской моделью},
    % Здесь указывается тип работы. Возможные значения:
    %   coursework - Курсовая работа
    %   diploma - Диплом специалиста
    %   master - Диплом магистра
    %   bachelor - Диплом бакалавра
    type               = {master},
    author             = {Тюляндин Иван Владимирович},
    supervisorPosition = {к.\,ф.-м.\,н., доцент кафедры информатики\\},
    supervisor         = {С.\,В. Григорьев},
    consultantPosition   = {к.\,ф.-м.\,н., ст.преп. кафедры современного программирования факультета МКН\\},
    consultant           = {Д.\,А. Березун},
    reviewerPosition   = {Санкт-Петербургский Политехнический Университет, старший преподаватель\\},
    reviewer           = {М. Х. Ахин}
%   university         = {Санкт-Петербургский Государственный Университет},
%   faculty            = {Математико-механический факультет},
%   city               = {Санкт-Петербург},
%   year               = {2013}
}
\filltitle{en}{
    chair              = {Software Engineering},
    title              = {Viterbi algorithm specialization with hidden Markov model},
    author             = {Ivan Tyulyandin},
    supervisorPosition = {Assistant Professor, PhD\\},
    supervisor         = {Semyon Grigorev},
    consultantPosition   = {Department of Mathematics and Computer Science, senior lecturer, PhD\\},
    consultant           = {Daniil Berezun},
    reviewerPosition   = {Peter the Great St.Petersburg Polytechnic University, senior lecturer\\},
    reviewer           = {Akhin Marat}
    % chairHeadPosition  = {professor},
    % chairHead          = {Christobal Junta},
}
\maketitle
\tableofcontents
% У введения нет номера главы

\section*{Введение}

Количество доступной информации в современном мире стремительно растет.
С целью повышения скорости обработки информации используются различные подходы, такие как вычислительные кластеры, новые алгоритмы и дополнительное оборудование, например, FPGA (ПЛИС) или GPGPU (графический процессор общего назначения). 
Однако на практике не всегда есть возможность увеличить имеющиеся вычислительные мощности, и в таком случае необходимо искать способы улучшения алгоритмов.

Одним из таких способов является описание и реализация отдельных шагов существующих алгоритмов  с использованием иных понятий и формализмов.
Для многих алгоритмов отдельные шаги могут быть описаны с использованием методов линейной алгебры, например, через операции над матрицами с переопределёнными операциями поэлементного сложения и умножения.
Если описать алгоритм методами линейной алгебры и запрограммировать его с использованием промышленных библиотек, то он может значительно превзойти по скорости исходную реализацию за счет возможности эффективно распараллеливания матричных операций.
Алгоритмы, которые выражены через операции и понятия из линейной алгебры, широко используются в различных областях, таких как машинное обучение~\cite{LA_ML}, компьютерное зрение~\cite{LA_CV}, статистика~\cite{LA_STAT}, анализ программ на логических языках программирования ~\cite{part_eval_logic}, теория графов~\cite{SuiteSparse} и многих других.

Другой способ улучшения алгоритмов основан на следующем наблюдении. 
Достаточно часто случается ситуация, когда часть параметров алгоритма не меняется от запуска к запуску, т.е. они зафиксированы в течение некоторого значительного промежутка времени.
Зная фактические значения этих параметров, можно оптимизировать их использование в алгоритме. 
Таким образом, можно получить новый алгоритм, в котором вычисления, зависящие только от зафиксированных параметров, уже выполнены.
Результат выполнения нового алгоритма с оставшимися параметрами должен быть семантически эквивалентен результату выполнения исходного алгоритма на соответствующих данных. 
В сравнении с исходным алгоритмом, при повторных запусках новый алгоритм не выполняет те вычисления, которые зависят от зафиксированных параметров. 
Эти вычисления сделаны и сохранены на стадии генерации нового алгоритма.
Такая техника преобразования алгоритмов известна как “специализация”, или “частичное вычисление”, а программа, генерирующая новый алгоритм, называется “специализатор”~\cite{Jones_spec}.
На текущий момент вопрос о границах применимости специализации к алгоритмам, выраженных с помощью методов линейной алгебры, до конца не исследован.

В данной работе будет рассмотрена специализация известного алгоритма Витерби~\cite{Viterbi}, выраженного в терминах линейной алгебры~\cite{LA_Viterbi}. 
Этот алгоритм используется в биоинформатике~\cite{cudampf}, при распознавании речи~\cite{Rabiner_VA} и в финансовых расчетах~\cite{Viterbi_credit}. 
У него имеется два входных параметра: скрытая марковская модель (далее --- СММ)~\cite{Eddy_HMM} и последовательность наблюдений. 
Задачей алгоритма Витерби является вычислить  вероятность того, что последовательность наблюдений была сгенерирована именно с помощью данной СММ. 
Основная часть алгоритма Витерби существенно зависит от СММ, и в то же время на практике, как правило, используется какая-то одна марковская модель для анализа значительного количества последовательностей наблюдений. 
Следовательно, если  специализировать алгоритм Витерби скрытой марковской моделью, то это может дать значительный прирост производительности.
\section{Постановка задачи}
Цель данной работы --- исследовать применимость специализации
алгоритма Витерби скрытой марковской моделью.
Если специализированная версия окажется производительнее, 
то это может ускорить исследования, проводимые биологами по 
изучению функциональности протеинов.

Были поставлены следующие задачи:
\begin{itemize}
	\item сделать обзор предметной области и существующих решений задачи 
		гомологичности;
	\item реализовать неспециализированный алгоритм Витерби;
	\item написать специализатор алгоритма Витерби на СММ;
	\item провести сравнительный анализ специализированной программы с
		неспециализированной версией и существующими решениями.
\end{itemize}

\section{Обзор}
В разделе сделан обзор предметной области.
Описаны скрытые марковские модели и алгоритм Витерби для работы с ними.
Приведены существующие проекты, решающие задачу гомологичности с применением скрытых марковских моделей.
Дано определение специализации.


\subsection{Скрытые марковские модели}
\label{lab:HMM}
\emph{Скрытая марковская модель}~\cite{Eddy_CHMM} (СММ) 
является дискретным вероятностным автоматом.
Модель имеет следующие параметры: множество состояний $St_{1..N}$, 
каждое из которых может создать событие из множества
$Obs_{1..K}$, вероятности $Pr\_b_{1..N}$ состояний быть
начальными, матрица $Tr$ вероятностей переходов между состояниями размера $N
\times N$ и матрица $Em$ вероятностей наблюдения события в определённом 
состоянии размера ${N \times K}$.

\subsection{Алгоритм Витерби}
\label{lab:Viterbi}
\emph{Алгоритм Витерби}~\cite{Viterbi}
(листинг~\ref{Viterbi}) считает вероятность нахождения в 
каждом состоянии СММ при условии того, что мы наблюдали последовательность событий \emph{O}.

\subsubsection{Описание методами динамического программирования}
\label{lab:dyn_Viterbi}
\begin{lstlisting}[caption=Псевдокод алгоритма Витерби, label=Viterbi, escapeinside={(*}{*)}]
function Viterbi(St, Obs, Pr_b, Tr, Em, O)
	T = length(O)
	Dp[T][N]

	for j = 1..N
		Dp[1][j] = Pr_b[j] * Em[j][O[1]]
	
	for i = 2..T
		for j = 1..N
			Dp[i][j] = 
					(*$\max_{x = 1..N}$*)(Dp[i-1][x] * Tr[x][j] * Em[j][O[i]])

	return Dp[T]
\end{lstlisting}


\subsubsection{Описание методами линейной алгебры}
\label{lab:LA_Viterbi}
Алгоритм Витерби выражается в терминах матричных 
операций из линейной алгебры~\cite{LA_Viterbi}.
Таким образом, этот алгоритм может быть реализован с 
использованием высокопроизводительных библиотек линейной 
алгебры, таких как
\name{SuiteSparse:GraphBLAS}~\cite{SuiteSparse}.

Ключевой идеей является определение специальной 
алгебраической структуры полукольцо \emph{Min\_plus} 
с операциями поэлементного сложения и умножения.
Элементы полукольца будут описывать вероятности в матрицах с
помощью дробных чисел.
Операция сложения имеет семантику взятия 
минимума из двух чисел, операция умножения --- семантику 
сложения чисел.
Нейтральным элементом по сложению будет $+\infty$, 
а по умножению 0.
Ниже приведен пример умножения матрицы на столбец 
с использованием полукольца \emph{Min\_plus}.
\[
  \begin{pmatrix}
    0 & 1 \\
    +\infty & 2
  \end{pmatrix}
  \begin{pmatrix}
    3 \\
    4
  \end{pmatrix}
  =
  \begin{pmatrix}
    min(0 + 3, 1 + 4) \\
    min(+\infty + 3, 2 + 4)
  \end{pmatrix}
  =
  \begin{pmatrix}
    3 \\
    6
  \end{pmatrix}
\]

Ко всем вероятностям в СММ применяется следующее
преобразование: отрицательный двоичный логарифм 
исходной вероятности.
Например, вероятность 0.5
будет представлена как $-1 * log_2(0.5) = 1$.
Это делается для сохранения точности расчетов.
Далее такая вероятность будет называться \emph{преобразованной}.

Для каждого события \emph{o} из множества \emph{Obs} 
определяем диагональную матрицу $P(o)$ размера $N \times N$.
Выражение $p_s(o)$ обозначает преобразованную вероятность наблюдать событие \emph{o} в состоянии \emph{s}.
\[
  P(o) =
  \begin{pmatrix}
    p_{1}(o) & \hdots & +\infty \\
    \vdots & \ddots & \vdots\\
    +\infty & \hdots & p_{N}(o)
  \end{pmatrix}
\]
Начало алгоритма Витерби --- это обработка первого события из 
последовательности \emph{O}.
В столбце \emph{Pr\_b} хранятся преобразованные вероятности 
состояний из СММ быть начальными.
Символ $\times$ обозначает умножение матриц с использованием 
полукольца \emph{Min\_plus}.
\[Probs_{1} = P(O[1]) \times Pr\_b\]
Далее вычисляются преобразованные вероятности для всех 
оставшихся событий из \emph{O}.
Матрица \emph{T} хранит преобразованные вероятности 
переходов из состояния в состояние:
\[Probs_{t} = P(O[t]) \times Tr^{T} \times Probs_{t - 1}\]
После выполнения всех шагов алгоритма, в столбце 
\emph{Probs\textsubscript{длина\_O}}
будут находиться преобразованные вероятности быть в 
определённом состоянии СММ при условии наблюдения 
последовательности событий \emph{O}.


\subsection{Существующие реализации алгоритма Витерби}
В данном разделе рассмотрены существующие высокопроизводительные реализации алгоритма Витерби, которые используются на практике.

\subsubsection{\name{HMMer}}
\name{HMMer}~\cite{HMMer} используется для поиска в базах данных 
последовательностей гомологов исследуемых протеинов, а также для создания 
профилей семейств протеинов.
Это проект с открытым исходным кодом.
Написан на языке \name{C} с возможностью использовать \name{SIMD} инструкции 
процессора.
Применяется во многих базах данных, таких как \name{Pfam}~\cite{Pfam}.

Авторами проекта были предложены вероятностные фильтры, которые используют 
\name{P7Viterbi} с частью удалённых состояний.
Применение фильтров позволяет ускорить обработку данных за счет уменьшения
вычислений в алгоритме Витерби.
Один из таких фильтров --- \name{MSV} (Multiple Segment 
Viterbi)~\cite{MSV_Eddy}.
Он моделирует последовательность из одной или более частей, 
внутри которых аминокислоты не могут быть удалены или 
вставлены.
Соответственно, веса перехода в состояния с префиксами D и I 
считаются нулевыми (рис.~\ref{MSV_example}).
\begin{figure}[t]
  \centering
  \includegraphics[width=\columnwidth]{MSV.png}
  \caption{Пример СММ вероятностного фильтра MSV~\cite{MSV_Eddy}}
  \label{MSV_example}
\end{figure}

\subsubsection{\name{CUDAMPF}}
В проекте \name{CUDAMPF}~\cite{cudampf} запрограммированы вероятностные фильтры из 
\name{HMMer} с использованием \name{CUDA}.
Код предназначен для видеокарты \name{NVIDIA} \name{Tesla K40} архитектуры
\name{Kepler}.
Проект рассчитан на определение гомологичности одновременно для множества 
протеинов.

Авторы предлагают четыре уровня параллелизма.
Первые три основаны на логическом параллелизме по данным.
Четвертый уровень использует \name{SIMD} инструкции вычислителей видеокарты.
Разделение данных по уровням позволило добиться ускорения в 23,1 раз при работе 
с фильтром \name{MSV} и в 11,6 раз с \name{P7Viterbi} по сравнению с
\name{HMMer}.

Несмотря на то, что авторами заявлена корректность реализации, в коде некорректно
вычисляется состояние E при обработке \name{P7Viterbi} и \name{MSV}.
Для этого состояния необходимо искать максимум на текущем шаге из состояний с 
префиксом M.
В исходном коде \name{CUDAMPF} переменная, хранящая максимум, 
не защищена от одновременной записи двумя или более потоками.

\subsection{Специализация}
При имеющейся программе $P$ и всех её входных параметрах 
$in$, можно получить результат выполнения $P$ на $in$.
При наличии только части параметров $in_1$ из $in$, 
\emph{специализатор} должен выполнить
вычисления и оптимизации кода в $P$, зависящие от $in_1$, 
а затем сгенерировать программу $P_{in1}$, 
которая будет принимать оставшиеся параметры $in_2$ и 
выполнять последующие вычисления (рис. \ref{spec}).
Параметры $in_1$ называются статическими, 
а $in_2$ --- динамическими.
Результат выполнения $P_{in1}$ на $in_2$ должен быть равен 
результату выполнения $P$ с параметрами $in$.
Ожидается, что специализированная программа будет 
производительнее неспециализированной версии за счет 
уменьшения вычислений или более оптимального использования 
оборудования, на котором выполняется программа.
Распространённой проблемой на практике является замедление 
производительности $P_{in1}$ из-за большого объема 
сгенерированного кода.
\begin{figure}
  \centering
  \includegraphics[width=\columnwidth]{spec.png}
  \caption{Специализация~\cite{Jones_spec}}
  \label{spec}
\end{figure}

Специализация была успешно применена в обработке 
графики~\cite{RT_spec}, обработке запросов к базам 
данных~\cite{SQL_spec} и поиску подстроки в строке на 
GPGPU~\cite{part_eval_GPU}.
Более подробно о специализации можно узнать в книге Джонса, 
Гомарда и Сестофта \cite{Jones_spec}.
\section{Специализация алгоритма Витерби}
В этой главе описаны подходы, которые были применены для 
специализации алгоритма Витерби.

\subsection{Описание форматов входных параметров}
Для экспериментов был создан формат описания данных в скрытой марковской модели.

\subsection{Специализация с использованием матричных\\ операций}
В статье Э. Теодосиса и П. Марагоса~\cite{LA_Viterbi} описан 
вариант представления алгоритма Витерби через матричные 
операции.
Он предназначен для работы со СММ из раздела 
\ref{lab:HMM} и рассмотрен в подразделе 
\ref{lab:LA_Viterbi}.

Рассмотрим операции алгоритма Витерби и выделим те части, которые можно специализировать, с учетом того, что 
СММ --- статический параметр.
Последовательность наблюдений $Obs$, которая является 
динамическим параметром, индексируется от 1 до $lo$ 
включительно, где $lo$ --- это длина последовательности.
Начальный шаг --- это обработка первого наблюдения из 
последовательности событий $Obs$.
\[Probs_{1} = P(Obs[1]) \times B\]
В СММ записано множество возможных наблюдений \emph{O}.
Матрицы $P(o)$ с преобразованными вероятностями для каждого 
наблюдения $o$ и столбец преобразованных вероятностей 
$B$ состояний быть начальным могут быть
получены из данных СММ.
Следовательно, можно заранее вычислить всевозможные варианты 
столбца $Probs_{1}$ как $K$ матриц $PB(o)$. 
\[PB(o) = P(o) \times B \;\;\; \forall o \in O\]
Далее в неспециализированной версии обрабатывается оставшаяся 
часть последовательности $Obs$.
\[Probs_{t} = P(Obs[t]) \times T^{\top} \times Probs_{t - 1}\]
Матрица переходов $T$ также хранится в СММ.
Это значит, что умножение матрицы $P(o)$ на $T^{T}$ 
может быть посчитано для любого наблюдения $o$ из множества $O$.
\[PT(o) = P(o) \times T^{\top} \;\;\; \forall o \in O\]
Все предпосчитанные матрицы сохраняются в памяти для 
дальнейшего переиспользования.
Псевдокод специализированного алгоритма Витерби представлен 
на листинге~\ref{Viterbi_1}.
\begin{lstlisting}[caption={Псевдокод алгоритма Витерби первого уровня специализации}, label=Viterbi_1, escapeinside={(*}{*)}]
HMM // (*\color{codegreen}{Произвольная СММ}*)
PB[HMM.K] // (*\color{codegreen}{$P(o) \times B$}*)
PT[HMM.K] // (*\color{codegreen}{$P(o) \times T^{\top}$}*)

function spec_Viterbi()
	for i = 1..HMM.K
		PB[i] = P(HMM.O[i]) (*$\times$*) HMM.B)
		PT[i] = P(HMM.O[i]) (*$\times$*) (HMM.T)(*$^{\top}$*)

function Viterbi(Obs)
	lo = length(Obs)
	Probs[1][HMM.K]

	Probs = PB(Obs[1])
	
	for i = 2..lo
		Probs = PT(Obs[i]) (*$\times$*) Probs
		
	return Probs
\end{lstlisting}

Введем понятие \emph{уровня специализации} --- это количество 
наблюдений, которое обрабатывается за одно умножение матриц 
при вычислениях со второго и последующих наблюдений.
Например, на листинге~\ref{Viterbi_1} уровень специализации 
равен одному, так как на строке 17 происходит обработка 
только одного наблюдения.

Далее можно воспользоваться тем фактом, что умножение матриц 
является ассоциативной операцией.
Это позволяет увеличить уровень специализации и тем самым 
сократить количество матричных умножений.
В формуле~\ref{lvl_2} показано, как обработать наблюдения $o_{t}$ и $o_{t-1}$ при условии, 
что $Probs_{t-2}$ известно.
\begin{align}
  \mathit{Probs}_{t} &= \mathit{PT}(\mathit{o}_{t}) \times \mathit{Probs}_{t-1}\nonumber\\
  &= \mathit{PT}(\mathit{o}_{t}) \times (\mathit{PT}(\mathit{o}_{t-1}) \times \mathit{Probs}_{t-2}) \nonumber\\
  & =(\mathit{PT}(\mathit{o}_{t}) \times \mathit{PT}(\mathit{o}_{t-1})) \times \mathit{Probs}_{t-2}
\label{lvl_2}
\end{align}
Результат умножения матриц $PT(o_t)$ и $PT(o_{t-1})$
можно получить, взяв данные из СММ.
Такой подход дает основу для повышения уровня специализации, 
который ограничен лишь количеством имеющейся памяти для хранения предпосчитанных матриц.
На формуле~\ref{lvl_3} представлен способ для обработки трех 
наблюдений.
Так как произведение $\mathit{PT}(o_t) \times \mathit{PT}(o_{t-1}) \times \mathit{PT}(o_{t-2})$
может быть вычислено на стадии специализации, необходимо 
только одно умножение матриц.
Та же самая идея может быть использована, чтобы получить 
четверый, пятый, и т.д. уровни специализации.
\begin{align}
  \mathit{Probs}_{t} = \mathit{PT}(o_t) \times \mathit{PT}(o_{t-1}) \times \mathit{PT}(o_{t-2}) \times \mathit{Probs}_{t - 3} 
\label{lvl_3}
\end{align}
Псевдокод алгоритма Витерби с произвольным уровнем 
специализации представлен на листинге~\ref{lvl_any}.
Для того, что получить нужный уровень специализации $M$, 
необходимо посчитать и сохранить произведение всевозможных 
комбинаций $M$ матриц $PT(o)$.
\begin{lstlisting}[caption={Псевдокод алгоритма Витерби произвольного уровня специализации}, label=lvl_any, escapeinside={(*}{*)}]
HMM
PB[HMM.K]
PT[HMM.K]
level
// obs_lvl_handlers (*\color{codegreen}{хранит всевозможные комбинации произведений level матриц из PT}*)
obs_lvl_handlers[HMM.K(*$^{level}$*)]

function spec_Viterbi()
	for i = 1..HMM.K
		PB[i] = P(HMM.O[i]) (*$\times$*) HMM.B)
		PT[i] = P(HMM.O[i]) (*$\times$*) (HMM.T)(*$^{\top}$*)
	calculate_combinations(obs_lvl_handlers, level, PT)

function Viterbi(Obs)
	// (*\color{codegreen}{Обработка первого наблюдения}*)
	Probs = PB[Obs[1]]

	lo = length(Obs)
	i = 2

	// (*\color{codegreen}{Пока количество необработанных наблюдений больше или равно level}*)
	while (lo - i) >= level)
		// (*\color{codegreen}{Ищем матрицу для обработки следующих level наблюдений}*)
		handler = obs_lvl_handlers.find(Obs[i:i+lvl])
		Probs = handler (*$\times$*) Probs
		i = i + level
	
	// (*\color{codegreen}{Количество необработанных наблюдений меньше, чем level}*)
	for (; i < lo; i = i + 1)
		Probs = PT[Obs[i]] (*$\times$*) Probs

	return Probs
\end{lstlisting}
В неспециализированной версии алгоритма Витерби необходимо 
выполнить $1 + 2 * (lo - 1)$ матричных умножений, где $lo$ 
--- это длина последовательности $Obs$.
При специализации уровня $M$ количество 
матричных умножений уменьшается, в таком случае нужно 
вычислить $\mathit{(lo - 1) / M + (lo - 1)\ mod\ M}$ 
произведений и выделить дополнительную память для хранения 
матриц $PT$ и $PB$ (это $2 * K$ матриц $N \times N$) и $K^{M}$ матриц $N 
\times N$ для обработки последовательности наблюдений 
размером $M$.

Таким образом, при специализации алгоритма Витерби в 
терминах линейной алгебры возможно значительное сокращение 
количества матричных операций в сравнении с 
неспециализированной версией, но при этом растет количество 
требуемой памяти.

\subsection{Реализация и тестирование  корректности}
Матрицы, которые описывают СММ, во многих случаях можно 
считать разреженными, то есть количество не нулевых элементов 
гораздо меньше, чем элементов всего.
Для работы с разреженными матрицами сообществом был создан 
стандарт \name{GraphBLAS}~\cite{GraphBLAS}.
Для проведения экспериментов по специализации алгоритма 
Витерби в терминах линейной алгебры была взята библиотека 
\name{SuiteSparse:GraphBLAS}~\cite{SuiteSparse}, 
которая де-факто считается самой производительной и является 
референсной реализацией стандарта \name{GraphBLAS}.

Специализатор считывает СММ, выполняет умножения матриц, 
которые зависят от статических данных из СММ, и результаты 
сохраняет в памяти, создавая специализированную функцию.
Далее, при вызове этой функции в зависимости от наблюдаемых 
событий, подставляются предпосчитанные матрицы.
Исходный код этого специализатора доступен по ссылке 
\href{https://github.com/IvanTyulyandin/Lin_alg_Viterbi}
{github.com/IvanTyulyandin/Lin\_alg\_Viterbi}.
\section{Эксперименты}
В данной главе описаны эксперименты по сравнению 
производительности специализаторов реализующих различные 
подходы.

Измерения выполнялись на компьютере с операционной системой 
\name{Ubuntu 20.04}, процессором \name{Intel Core} i7-4790, 
видеокартой \name{NVIDIA GeForce GTX 1030} и 32 Гб 
оперативной памяти.
Во время замера брались СММ с различным количеством 
состояний, далее запускалась соответствующая реализация 
алгоритма Витерби на 3 последовательностях по 7000 
наблюдений, сгенерированных случайным образом.
В качестве результата бралось лучшее время из 3 
запусков.

\subsection{Описание набора данных и оборудования}
В качестве тестового набора был взяты данные из репозитория  
проекта \name{CUDAMPF}~\cite{cudampf}.
Замеры выполнялись на видеокарте \name{NVIDIA GeForce GTX 
1030}.

\subsection{Сравнение производительности}
Для замера этого специализатора был написан генератор СММ.
Его задача создавать СММ по количеству состояний, переходов 
между ними и количеству наблюдений.
Были сгенерированы СММ с такими же характеристиками,
как в предыдущем разделе.
На текущий момент \name{SuiteSparse:GraphBLAS} может 
выполняться только на процессоре, измерения проводились на 
\name{Intel Core} i7-4790 с частотой 3.60 GHz.
\begin{figure}[h]
\centering
	    \begin{tikzpicture}
        \begin{axis}[
	        title={SuiteSparse:GraphBLAS, Intel Core i7-4790},
            axis x line=bottom,
            axis y line=left,
            xlabel={Кол-во состояний СММ},
            ylabel={Время, мсек},
            legend pos=south east]
            \addplot[mark=square,red,thick] table[x=states,y=non_spec] {GraphBLAS_bench.dat};
            \addplot[mark=square,blue,thick] table[x=states,y=spec] {GraphBLAS_bench.dat};
            \legend{Обыч., Спец.}
        \end{axis}
	\end{tikzpicture}
\caption{Измерение производительности специализатора алгоритма Витерби в терминах линейной алгебры}	
\label{LA_bench}
\end{figure}

По графику на рисунке~\ref{LA_bench} можно сделать следующие 
выводы.
Во-первых, специализированная версия дает повышение 
производительности примерно на 20\% процентов.
Во-вторых, в некоторых случаях скорость выполнения алгоритма 
выше на процессоре, чем на видеокарте, это связано с 
отсутствием накладных расходов.

\newpage
\section{Заключение}
В ходе работы были получены результаты, перечисленные ниже.
\begin{itemize}
	\item Выполнен обзор предметной области:
		\begin{itemize}
			\item рассмотрен алгоритм Витерби и его реализации \name{HMMER} и \name{CUDAMPF};
			\item изучена техника специализации.
		\end{itemize}
	\item Реализованы и протестированы две реализации специализированного алгоритма Витерби, описанного с помощью алгебраической структуры полукольцо \emph{Min-plus} и матричных операций:
		\begin{itemize}
			\item с использованием библиотеки \name{Sui\-te\-Spar\-se:Graph\-BLAS} для выполнения на центральном процессоре;
			\item с использованием библиотеки \name{cuASR} для выполнения на графических процессорах общего назначения.
		\end{itemize}
	\item Проведены эксперименты на данных из репозитория \name{CUDAMPF}. 
Установлено, что специализированный алгоритм Витерби производительнее неспециализированной версии.
\end{itemize}

Статья \emph{Viterbi Algorithm Specialization Using Linear 
Algebra} была принята на конференции \name{SEIM 2021}, а 
также проведена публичная презентация результатов~\cite{paper}.

Исходный код реализаций алгоритма Витерби с использованием методов линейной алгебры доступен по ссылке~\cite{repo}.

Исходя из вышеперечисленных результатов, специализация 
алгоритма Витерби, выраженного методами линейной алгебры, 
скрытой марковской моделью может дать значительный прирост 
производительности.


\setmonofont[Mapping=tex-text]{CMU Typewriter Text}
\bibliographystyle{ugost2008ls}
\bibliography{diploma.bib}
\end{document}
