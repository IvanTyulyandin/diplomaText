\section{Специализация алгоритма Витерби}
В этой главе описаны подходы, которые были применены для 
специализации алгоритма Витерби.

\subsection{Описание форматов входных параметров}
Для экспериментов был создан формат описания данных в скрытой марковской модели.

\subsection{Специализация с использованием матричных операций}
В статье Э. Теодосиса и П. Марагоса~\cite{LA_Viterbi} описан 
вариант представления алгоритма Витерби через матричные 
операции.
Он предназначен для работы со СММ из раздела 
\ref{HMM_Vit} и рассмотрен в параграфе 
\ref{LA_Viterbi_review}.

Рассмотрим операции алгоритма и выделим те части, 
которые можно специализировать.
Начальный шаг --- это обработка первого события из 
последовательности событий \emph{O}.
\[Probs_{1} = P(O[1]) \times Pr\_b\]
В СММ записано множество наблюдаемых событий \emph{Obs}.
Матрицы $P(o)$ с преобразованными вероятностями для каждого 
события $o$ и столбец преобразованных вероятностей 
\emph{Pr\_b} состояний быть начальным могут быть 
получены из данных СММ.
Следовательно, можно заранее вычислить всевозможные варианты 
столбца $Probs_{1}$. 
Далее обрабатывается оставшаяся часть последовательности 
\emph{O}.
\[Probs_{t} = P(O[t]) \times Tr^{T} \times Probs_{t - 1}\]
Матрица переходов \emph{Tr} хранится в СММ.
Это значит, что умножение матрицы $P(o)$ на $Tr^{T}$ также 
может быть посчитано.

Таким образом, при специализации алгоритма Витерби в 
терминах линейной алгебры возможно сокращение количества 
матричных операций почти в два раза в сравнении с 
неспециализированной версией.

\subsection{Реализация и тестирование  корректности}
Матрицы, которые описывают СММ, во многих случаях можно 
считать разреженными, то есть количество не нулевых элементов 
гораздо меньше, чем элементов всего.
Для работы с разреженными матрицами сообществом был создан 
стандарт \name{GraphBLAS}~\cite{GraphBLAS}.
Для проведения экспериментов по специализации алгоритма 
Витерби в терминах линейной алгебры была взята библиотека 
\name{SuiteSparse:GraphBLAS}~\cite{SuiteSparse}, 
которая де-факто считается самой производительной и является 
референсной реализацией стандарта \name{GraphBLAS}.

Специализатор считывает СММ, выполняет умножения матриц, 
которые зависят от статических данных из СММ, и результаты 
сохраняет в памяти, создавая специализированную функцию.
Далее, при вызове этой функции в зависимости от наблюдаемых 
событий, подставляются предпосчитанные матрицы.
Исходный код этого специализатора доступен по ссылке 
\href{https://github.com/IvanTyulyandin/Lin_alg_Viterbi}
{github.com/IvanTyulyandin/Lin\_alg\_Viterbi}.