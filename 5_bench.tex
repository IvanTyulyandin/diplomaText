\section{Эксперименты}
В данной главе описаны эксперименты по сравнению 
производительности неспециализированного алгоритма Витерби 
против специализированного алгоритма Витерби первого, второго 
и третьего уровня, а также по сравнению с реализацией из существующего решения \name{CUDAMPF} из 
подраздела~\ref{lab:CUDAMPF}.
Существующее решение \name{HMMer} из подраздела~\ref{lab:HMMer} не измерялось, так как согласно статье~\cite{cudampf} \name{CUDAMPF} превосходит \name{HMMer} более чем в 20 раз.

\subsection{Описание набора данных и оборудования}
Эксперименты выполнялись на рабочей станции с ОС Ubuntu 
20.04, процессором \name{AMD Ryzen} 1600, видеокартой \name{NVI\-DIA GeForce GTX 1070} и 16 Гб 
оперативной памяти.

В качестве тестового набора были взяты СММ из репозитория  
проекта \name{CUDAMPF}.
Это 24 так называемых \emph{молчаливых СММ}~\cite{silentHMM}, 
размером от 100 до 2405 состояний.
Каждая из этих молчаливых СММ описывает вероятностный фильтр \name{MSV}, то есть СММ, специфичную для бионформатики.
Эти СММ отличаются от СММ, описанных в разделе~\ref{lab:HMM} 
тем, что состояния могут не создавать наблюдения.
Из-за этого отличия алгоритм Витерби из 
раздела~\ref{lab:Viterbi} не будет работать корректно.
Реализация алгоритма Витерби в \name{CUDAMPF} адаптирована для работы с \name{MSV}.
Данные молчаливые СММ из репозитория \name{CUDAMPF} можно использовать 
для моделирования вычислительной нагрузки.
Чтобы ранее описанные варианты алгоритма Витерби могли 
работать с данными из молчаливой СММ, был разработан 
конвертер, который по молчаливой СММ создает СММ с теми же 
состояниями и, по возможности, сохраняет переходы между ними, 
либо добавляет новые переходы, если состояние не создавало 
наблюдений.
Количество возможных наблюдений, т.е. $K$, равно 20.

Были подготовлены следующие наборы последовательностей наблюдений:
\begin{itemize}
	\item 3 последовательности по 3500 наблюдений;
	\item 3 последовательности по 7000 наблюдений;
	\item 16 последовательностей размером от 38 до 7096;
	\item 50 последовательностей по 3500 наблюдений.
\end{itemize}
Первый, второй и четвертый наборы были искусственно сгенерированы, в то время 
как третий был взят из базы данных протеинов \name{PFAM} и 
приведен к формату \emph{ess}, который описан в 
разделе~\ref{lab:formats}.


\subsection{Анализ результатов}
Для получения времени выполнения каждая реализация алгоритма 
Витерби запускалась 10 раз, и из этих результатов бралась 
медиана.
Измерялось время обработки всего набора последовательностей при зафиксированной СММ конкретной реализацией.
Если реализация специализированная, то также измерялось время, необходимое на выполнение специализации.

Для реализации с использованием \name{SuiteSparse:GraphBLAS} были получены следующие результаты, представленные на рисунках~\ref{3500_SS},~\ref{7000_SS},~\ref{RW_SS},
~\ref{50_SS},~\ref{Spec_time_SS} и в таблице~\ref{runtime}.
\begin{figure}[!b]
\centering
	\begin{tikzpicture}
		\begin{axis}[
     	  title={},
          axis x line=bottom,
          axis y line=left,
          xlabel={Кол-во состояний в СММ},
          ylabel={Время, мс},
          legend pos=outer north east]
        	\addplot[mark=square,red,thick] table[x=States,y=CUDAMPF] {bench_CUDAMPF_emit_3_3500.dat};
        	\addplot[mark=square,blue,thick] table[x=States,y=GraphBLAS] {Viterbi_bench_emit_3_3500_20.dat};
        	\addplot[mark=square,green,thick] table[x=States,y=GraphBLAS_spec_1] {Viterbi_spec_bench_emit_3_3500_20.dat};
        	\addplot[mark=square,magenta,thick] table[x=States,y=GraphBLAS_spec_2] {Viterbi_spec_bench_emit_3_3500_20.dat};
        	\addplot[mark=square,black,thick] table[x=States,y=GraphBLAS_spec_3] {Viterbi_spec_bench_emit_3_3500_20.dat};
        	\legend{CUDAMPF, неспец., ур. 1 спец., ур. 2 спец., ур. 3 спец.}
      		\end{axis}
    \end{tikzpicture}
    \caption{GraphBLAS, 3 x 3500 наблюдений, меньше --- лучше}
\label{3500_SS}
\end{figure}

\begin{figure}[h!]
\centering
	\begin{tikzpicture}
		\begin{axis}[
	      title={},
          axis x line=bottom,
          axis y line=left,
          xlabel={Кол-во состояний в СММ},
          ylabel={Время, мс},
          legend pos=outer north east]
        	\addplot[mark=square,red,thick] table[x=States,y=CUDAMPF] {bench_CUDAMPF_emit_3_7000.dat};
        	\addplot[mark=square,blue,thick] table[x=States,y=GraphBLAS] {Viterbi_bench_emit_3_7000_20.dat};
        	\addplot[mark=square,green,thick] table[x=States,y=GraphBLAS_spec_1] {Viterbi_spec_bench_emit_3_7000_20.dat};
        	\addplot[mark=square,magenta,thick] table[x=States,y=GraphBLAS_spec_2] {Viterbi_spec_bench_emit_3_7000_20.dat};
        	\addplot[mark=square,black,thick] table[x=States,y=GraphBLAS_spec_3] {Viterbi_spec_bench_emit_3_7000_20.dat};
    	    \legend{CUDAMPF, неспец., ур. 1 спец., ур. 2 спец., ур. 3 спец.}
      		\end{axis}
    \end{tikzpicture}
    \caption{GraphBLAS, 3 х 7000 наблюдений, меньше --- лучше}
\label{7000_SS}
\end{figure}

\begin{figure}[h!]
\centering
	\begin{tikzpicture}
		\begin{axis}[
	  	  title={},
          axis x line=bottom,
          axis y line=left,
          xlabel={Кол-во состояний в СММ},
          ylabel={Время, мс},
          legend pos=outer north east]
        	\addplot[mark=square,red,thick] table[x=States,y=CUDAMPF] {bench_CUDAMPF_covid.dat};
        	\addplot[mark=square,blue,thick] table[x=States,y=GraphBLAS] {Viterbi_bench_covid-19.dat};
        	\addplot[mark=square,green,thick] table[x=States,y=GraphBLAS_spec_1] {Viterbi_spec_bench_covid-19.dat};
        	\addplot[mark=square,magenta,thick] table[x=States,y=GraphBLAS_spec_2] {Viterbi_spec_bench_covid-19.dat};
        	\addplot[mark=square,black,thick] table[x=States,y=GraphBLAS_spec_3] {Viterbi_spec_bench_covid-19.dat};
      	    \legend{CUDAMPF, неспец., ур. 1 спец., ур. 2 спец., ур. 3 спец.}
      	\end{axis}
    \end{tikzpicture}
    \caption{GraphBLAS, набор данных из БД \name{PFAM}, меньше --- лучше}
\label{RW_SS}    
\end{figure}

\begin{figure}[h!]
\centering
	\begin{tikzpicture}
		\begin{axis}[
	      title={},
          axis x line=bottom,
          axis y line=left,
          xlabel={Кол-во состояний в СММ},
          ylabel={Время, мс},
          legend pos=outer north east]
        	\addplot[mark=square,red,thick] table[x=States,y=CUDAMPF] {bench_CUDAMPF_emit_50_3500.dat};
        	\addplot[mark=square,blue,thick] table[x=States,y=GraphBLAS] {Viterbi_bench_emit_50_3500_20.dat};
        	\addplot[mark=square,green,thick] table[x=States,y=GraphBLAS_spec_1] {Viterbi_spec_bench_emit_50_3500_20.dat};
        	\addplot[mark=square,magenta,thick] table[x=States,y=GraphBLAS_spec_2] {Viterbi_spec_bench_emit_50_3500_20.dat};
        	\addplot[mark=square,black,thick] table[x=States,y=GraphBLAS_spec_3] {Viterbi_spec_bench_emit_50_3500_20.dat};
    	    \legend{CUDAMPF, неспец., ур. 1 спец., ур. 2 спец., ур. 3 спец.}
      		\end{axis}
    \end{tikzpicture}
    \caption{GraphBLAS, 50 х 3500 наблюдений, меньше --- лучше}
\label{50_SS}
\end{figure}


\begin{figure}[h!]
\centering
	\begin{tikzpicture}
		\begin{axis}[
		  title={},
          axis x line=bottom,
          axis y line=left,
          xlabel={Кол-во состояний в СММ},
          ylabel={Время, мс},
          legend pos=outer north east]
        \addplot[mark=square,green,thick] table[x=States,y=GraphBLAS_spec_1_prep] {Viterbi_spec_bench_emit_3_7000_20.dat};
        \addplot[mark=square,magenta,thick] table[x=States,y=GraphBLAS_spec_2_prep] {Viterbi_spec_bench_emit_3_7000_20.dat};
        \addplot[mark=square,black,thick] table[x=States,y=GraphBLAS_spec_3_prep] {Viterbi_spec_bench_emit_3_7000_20.dat};                  				\legend{ур. 1 спец., ур. 2 спец., ур. 3 спец.}
      \end{axis}
    \end{tikzpicture}
    \caption{GraphBLAS, время на специализацию, меньше --- лучше}
\label{Spec_time_SS}
\end{figure}

Как можно видеть из графиков, с повышением уровня 
специализации обработка набора последовательностей наблюдений 
выполняется быстрее для всех наборов данных.
Стоит также отметить, что специализация третьего уровня либо 
превосходит, либо равна по производительности с реализацией 
из \name{CUDAMPF}.
Время, затраченное на специализацию, возрастает с повышением 
уровня специализации.
Не смотря на то, что для третьего уровня время специализации 
разительно отличается от первого и второго уровня, при 
анализе достаточно большого набора последовательностей это 
время будет незначительным, так как процедура специализации 
делается один раз перед запуском анализа набора.
Из этого можно сделать вывод, что уровень специализации нужно 
выбирать, основываясь на наборе данных.
Например, по информации из таблицы~\ref{runtime} можно 
сделать вывод, что для относительно небольших наборов данных 
самым эффективным является второй уровень специализации, а 
при достаточно большом количестве последовательностей 
оптимальным будет третий уровень специализации.
\begin{table}[t!]
  \centering
  \begin{tabular}{||c c c c c c||} 
    \hline
    & CUDAMPF & Initial & 1-level & 2-level & 3-level\\ [0.5ex] 
    \hline\hline
    3 x 3500 & 4854 & 8900 & 6194 & \textbf{3840} & 16576 \\ 
    \hline
    3 x 7000 & 9209 & 17009 & 12381 & \textbf{6945} & 18258 \\
    \hline
    Данные из \name{PFAM} & 8796 & 12874 & 9302 & \textbf{5370} & 17329 \\
    \hline
    50 x 3500 & 103036 & 144369 & 99726 & 52235 & \textbf{49572} \\
    \hline
  \end{tabular}
  \caption{GraphBLAS, общее время обработки с учетом затрат времени на специализацию, мс}
  \label{runtime}
\end{table}

При измерении специализированных реализаций с использованием библиотеки \name{CUSP} были получены следующие результаты, которые представлены на рисунках~\ref{3500_CUSP},~\ref{7000_CUSP},~\ref{RW_CUSP},~\ref{50_CUSP},~\ref{Spec_time_CUSP}, а также в таблице~\ref{runtime_CUSP}.
\begin{figure}[h!]
\centering
	\begin{tikzpicture}
		\begin{axis}[
     	  title={},
          axis x line=bottom,
          axis y line=left,
          xlabel={Кол-во состояний в СММ},
          ylabel={Время, мс},
          legend pos=outer north east]
        	\addplot[mark=square,red,thick] table[x=States,y=CUDAMPF] {bench_CUDAMPF_emit_3_3500.dat};
        	\addplot[mark=square,blue,thick] table[x=States,y=CUSP] {Viterbi_bench_emit_3_3500_20.dat};
        	\addplot[mark=square,green,thick] table[x=States,y=CUSP_spec_1] {Viterbi_spec_bench_emit_3_3500_20.dat};
        	\addplot[mark=square,magenta,thick] table[x=States,y=CUSP_spec_2] {Viterbi_spec_bench_emit_3_3500_20.dat};
        	\addplot[mark=square,black,thick] table[x=States,y=CUSP_spec_3] {Viterbi_spec_bench_emit_3_3500_20.dat};
        	\legend{CUDAMPF, неспец., ур. 1 спец., ур. 2 спец., ур. 3 спец.}
      		\end{axis}
    \end{tikzpicture}
    \caption{CUSP, 3 x 3500 наблюдений, меньше --- лучше}
\label{3500_CUSP}
\end{figure}

\begin{figure}[h!]
\centering
	\begin{tikzpicture}
		\begin{axis}[
	      title={},
          axis x line=bottom,
          axis y line=left,
          xlabel={Кол-во состояний в СММ},
          ylabel={Время, мс},
          legend pos=outer north east]
        	\addplot[mark=square,red,thick] table[x=States,y=CUDAMPF] {bench_CUDAMPF_emit_3_7000.dat};
        	\addplot[mark=square,blue,thick] table[x=States,y=CUSP] {Viterbi_bench_emit_3_7000_20.dat};
        	\addplot[mark=square,green,thick] table[x=States,y=CUSP_spec_1] {Viterbi_spec_bench_emit_3_7000_20.dat};
        	\addplot[mark=square,magenta,thick] table[x=States,y=CUSP_spec_2] {Viterbi_spec_bench_emit_3_7000_20.dat};
        	\addplot[mark=square,black,thick] table[x=States,y=CUSP_spec_3] {Viterbi_spec_bench_emit_3_7000_20.dat};
    	    \legend{CUDAMPF, неспец., ур. 1 спец., ур. 2 спец., ур. 3 спец.}
      		\end{axis}
    \end{tikzpicture}
    \caption{CUSP, 3 х 7000 наблюдений, меньше --- лучше}
\label{7000_CUSP}
\end{figure}

\begin{figure}[h!]
\centering
	\begin{tikzpicture}
		\begin{axis}[
	  	  title={},
          axis x line=bottom,
          axis y line=left,
          xlabel={Кол-во состояний в СММ},
          ylabel={Время, мс},
          legend pos=outer north east]
        	\addplot[mark=square,red,thick] table[x=States,y=CUDAMPF] {bench_CUDAMPF_covid.dat};
        	\addplot[mark=square,blue,thick] table[x=States,y=CUSP] {Viterbi_bench_covid-19.dat};
        	\addplot[mark=square,green,thick] table[x=States,y=CUSP_spec_1] {Viterbi_spec_bench_covid-19.dat};
        	\addplot[mark=square,magenta,thick] table[x=States,y=CUSP_spec_2] {Viterbi_spec_bench_covid-19.dat};
        	\addplot[mark=square,black,thick] table[x=States,y=CUSP_spec_3] {Viterbi_spec_bench_covid-19.dat};
      	    \legend{CUDAMPF, неспец., ур. 1 спец., ур. 2 спец., ур. 3 спец.}
      	\end{axis}
    \end{tikzpicture}
    \caption{CUSP, набор данных из БД \name{PFAM}, меньше --- лучше}
\label{RW_CUSP}    
\end{figure}

\begin{figure}[h!]
\centering
	\begin{tikzpicture}
		\begin{axis}[
     	  title={},
          axis x line=bottom,
          axis y line=left,
          xlabel={Кол-во состояний в СММ},
          ylabel={Время, мс},
          legend pos=outer north east]
        	\addplot[mark=square,red,thick] table[x=States,y=CUDAMPF] {bench_CUDAMPF_emit_50_3500.dat};
        	\addplot[mark=square,blue,thick] table[x=States,y=CUSP] {Viterbi_bench_emit_50_3500_20.dat};
        	\addplot[mark=square,green,thick] table[x=States,y=CUSP_spec_1] {Viterbi_spec_bench_emit_50_3500_20.dat};
        	\addplot[mark=square,magenta,thick] table[x=States,y=CUSP_spec_2] {Viterbi_spec_bench_emit_50_3500_20.dat};
        	\addplot[mark=square,black,thick] table[x=States,y=CUSP_spec_3] {Viterbi_spec_bench_emit_50_3500_20.dat};
        	\legend{CUDAMPF, неспец., ур. 1 спец., ур. 2 спец., ур. 3 спец.}
      		\end{axis}
    \end{tikzpicture}
    \caption{CUSP, 50 x 3500 наблюдений, меньше --- лучше}
\label{50_CUSP}
\end{figure}


\begin{figure}[h!]
\centering
	\begin{tikzpicture}
		\begin{axis}[
		  title={},
          axis x line=bottom,
          axis y line=left,
          xlabel={Кол-во состояний в СММ},
          ylabel={Время, мс},
          legend pos=outer north east]
        \addplot[mark=square,green,thick] table[x=States,y=CUSP_spec_1_prep] {Viterbi_spec_bench_emit_3_7000_20.dat};
        \addplot[mark=square,magenta,thick] table[x=States,y=CUSP_spec_2_prep] {Viterbi_spec_bench_emit_3_7000_20.dat};
        \addplot[mark=square,black,thick] table[x=States,y=CUSP_spec_3_prep] {Viterbi_spec_bench_emit_3_7000_20.dat};                  				\legend{ур. 1 спец., ур. 2 спец., ур. 3 спец.}
      \end{axis}
    \end{tikzpicture}
    \caption{CUSP, время на специализацию, меньше --- лучше}
\label{Spec_time_CUSP}
\end{figure}


\begin{table}[t!]
  \centering
  \begin{tabular}{||c c c c c c||} 
    \hline
    & CUDAMPF & Initial & 1-level & 2-level & 3-level\\ [0.5ex] 
    \hline\hline
    3 x 3500 & \textbf{4854} & 422804 & 229836 & 129472 & 307650\\ 
    \hline
    3 x 7000 & \textbf{9209} & 806734 & 441478 & 247698 & 391031 \\
    \hline
    Данные из \name{PFAM} & \textbf{8796} & 605574 & 333534 & 191327 & 353298 \\
    \hline
    50 x 3500 & \textbf{103036} & 6709372 & 3617766 & 1951912 & 1604179 \\
    \hline
  \end{tabular}
  \caption{CUSP, общее время обработки с учетом затрат времени на специализацию, мс}
  \label{runtime_CUSP}
\end{table}

Исходя из полученных результатов, можно сделать вывод, что 
как и в случае с реализациями с использованием 
\name{GraphBLAS}, при повышении уровня специализации 
снижается время обработки последовательности.
В конкретном случае, реализации оказались не такими 
производительными по сравнению с \name{CUDAMPF} и 
\name{GraphBLAS}.
Причиной этого являются особенности реализации библиотеки 
\name{CUSP}.

Анализируя данных экспериментов, стоит отметить, что 
специализированные версии производительнее, чем 
неспециализированные, то есть специализация дает прирост по 
скорости обработке последовательностей наблюдений за счет 
уменьшения количества матричных операций.
При повышении уровня специализации также наблюдается улучшение производительности.
При этом оптимальный уровень специализации зависит от набора 
данных, который необходимо обработать.

\newpage