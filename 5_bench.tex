\section{Эксперименты}
В данной главе описаны эксперименты по сравнению 
производительности специализаторов реализующих различные 
подходы.

Измерения выполнялись на компьютере с операционной системой 
\name{Ubuntu 20.04}, процессором \name{Intel Core} i7-4790, 
видеокартой \name{NVIDIA GeForce GTX 1030} и 32 Гб 
оперативной памяти.
Во время замера брались СММ с различным количеством 
состояний, далее запускалась соответствующая реализация 
алгоритма Витерби на 3 последовательностях по 7000 
наблюдений, сгенерированных случайным образом.
В качестве результата бралось лучшее время из 3 
запусков.

\subsection{Описание набора данных и оборудования}
В качестве тестового набора был взяты данные из репозитория  
проекта \name{CUDAMPF}~\cite{cudampf}.
Замеры выполнялись на видеокарте \name{NVIDIA GeForce GTX 
1030}.

\subsection{Сравнение производительности}
Для замера этого специализатора был написан генератор СММ.
Его задача создавать СММ по количеству состояний, переходов 
между ними и количеству наблюдений.
Были сгенерированы СММ с такими же характеристиками,
как в предыдущем разделе.
На текущий момент \name{SuiteSparse:GraphBLAS} может 
выполняться только на процессоре, измерения проводились на 
\name{Intel Core} i7-4790 с частотой 3.60 GHz.
\begin{figure}[h]
\centering
	    \begin{tikzpicture}
        \begin{axis}[
	        title={SuiteSparse:GraphBLAS, Intel Core i7-4790},
            axis x line=bottom,
            axis y line=left,
            xlabel={Кол-во состояний СММ},
            ylabel={Время, мсек},
            legend pos=south east]
            \addplot[mark=square,red,thick] table[x=states,y=non_spec] {GraphBLAS_bench.dat};
            \addplot[mark=square,blue,thick] table[x=states,y=spec] {GraphBLAS_bench.dat};
            \legend{Обыч., Спец.}
        \end{axis}
	\end{tikzpicture}
\caption{Измерение производительности специализатора алгоритма Витерби в терминах линейной алгебры}	
\label{LA_bench}
\end{figure}

По графику на рисунке~\ref{LA_bench} можно сделать следующие 
выводы.
Во-первых, специализированная версия дает повышение 
производительности примерно на 20\% процентов.
Во-вторых, в некоторых случаях скорость выполнения алгоритма 
выше на процессоре, чем на видеокарте, это связано с 
отсутствием накладных расходов.
