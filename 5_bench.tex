\section{Эксперименты}
В данной главе описаны эксперименты по сравнению 
производительности неспециализированного алгоритма Витерби 
против специализированного алгоритма Витерби первого, второго 
и третьего уровня, а также по сравнению с реализацией из существующего решения \name{CUDAMPF} из 
подраздела~\ref{lab:CUDAMPF}.
Существующее решение \name{HMMer} из подраздела~\ref{lab:HMMer} не измерялось, так как согласно статье~\cite{cudampf} \name{CUDAMPF} превосходит \name{HMMer} более чем в 20 раз.

\subsection{Описание набора данных и оборудования}
Эксперименты выполнялись на рабочей станции с ОС Ubuntu 
20.04, процессором \name{Intel Core} i7-4790 с частотой 3.60 
GHz, видеокартой \name{NVI\-DIA GeForce GTX 1030} и 32 Гб 
оперативной памяти.

В качестве тестового набора были взяты СММ из репозитория  
проекта \name{CUDAMPF}.
Это 24 так называемых \emph{молчаливых СММ}~\cite{silentHMM}, 
размером от 100 до 2405 состояний.
Каждая из этих молчаливых СММ описывает вероятностный фильтр \name{MSV}, то есть СММ, специфичную для бионформатики.
Эти СММ отличаются от СММ, описанных в разделе~\ref{lab:HMM} 
тем, что состояния могут не создавать наблюдения.
Из-за этого отличия алгоритм Витерби из 
раздела~\ref{lab:Viterbi} не будет работать корректно.
Реализация алгоритма Витерби в \name{CUDAMPF} адаптирована для работы с \name{MSV}.
Данные молчаливые СММ из репозитория \name{CUDAMPF} можно использовать 
для моделирования вычислительной нагрузки.
Чтобы ранее описанные варианты алгоритма Витерби могли 
работать с данными из молчаливой СММ, был разработан 
конвертер, который по молчаливой СММ создает СММ с теми же 
состояниями и, по возможности, сохраняет переходы между ними, 
либо добавляет новые переходы, если состояние не создавало 
наблюдений.
Количество возможных наблюдений, т.е. $K$, равно 20.

Были подготовлены следующие наборы последовательностей наблюдений:
\begin{itemize}
	\item 3 последовательности по 3500 наблюдений;
	\item 3 последовательности по 7000 наблюдений;
	\item 16 последовательностей размером от 38 до 7096.
\end{itemize}
Первые два набора были искусственно сгенерированы, в то время 
как третий был взят из базы данных протеинов \name{PFAM} и 
приведен к формату \emph{ess}, который описан в 
разделе~\ref{lab:formats}.


\subsection{Анализ результатов}
Для получения времени выполнения каждая реализация алгоритма 
Витерби запускалась 100 раз, и из этих результатов бралась 
медиана.
Измерялось время обработки всего набора последовательностей при зафиксированной СММ конкретной реализацией.
Если реализация специализированная, то также измерялось время, необходимое на выполнение специализации.

Для реализации с использованием \name{SuiteSparse:GraphBLAS} были получены следующие результаты, представленные на рисунках~\ref{3500_SS},~\ref{7000_SS},~\ref{RW_SS}, ~\ref{Spec_time_SS} и в таблице~\ref{runtime}.
\begin{figure}[!b]
\centering
	\begin{tikzpicture}
		\begin{axis}[
     	  title={},
          axis x line=bottom,
          axis y line=left,
          xlabel={Кол-во состояний в СММ},
          ylabel={Время, мс},
          legend pos=north west]
        	\addplot[mark=square,red,thick] table[x=states,y=cudampf] {3500_bench.dat};
        	\addplot[mark=square,blue,thick] table[x=states,y=ns] {3500_median_bench.dat};
        	\addplot[mark=square,green,thick] table[x=states,y=s1] {3500_median_bench.dat};
        	\addplot[mark=square,magenta,thick] table[x=states,y=s2] {3500_median_bench.dat};
        	\addplot[mark=square,black,thick] table[x=states,y=s3] {3500_median_bench.dat};
        	\legend{CUDAMPF, неспец., ур. 1 спец., ур. 2 спец., ур. 3 спец.}
      		\end{axis}
    \end{tikzpicture}
    \caption{3 x 3500 наблюдений, меньше --- лучше}
\label{3500_SS}
\end{figure}

\begin{figure}[h]
\centering
	\begin{tikzpicture}
		\begin{axis}[
	      title={},
          axis x line=bottom,
          axis y line=left,
          xlabel={Кол-во состояний в СММ},
          ylabel={Время, мс},
          legend pos=north west]
        	\addplot[mark=square,red,thick] table[x=states,y=cudampf] {7000_bench.dat};
        	\addplot[mark=square,blue,thick] table[x=states,y=ns] {7000_median_bench.dat};
        	\addplot[mark=square,green,thick] table[x=states,y=s1] {7000_median_bench.dat};
        	\addplot[mark=square,magenta,thick] table[x=states,y=s2] {7000_median_bench.dat};
        	\addplot[mark=square,black,thick] table[x=states,y=s3] {7000_median_bench.dat};
    	    \legend{CUDAMPF, неспец., ур. 1 спец., ур. 2 спец., ур. 3 спец.}
      		\end{axis}
    \end{tikzpicture}
    \caption{3 х 7000 наблюдений, меньше --- лучше}
\label{7000_SS}
\end{figure}

\begin{figure}[h]
\centering
	\begin{tikzpicture}
		\begin{axis}[
	  	  title={},
          axis x line=bottom,
          axis y line=left,
          xlabel={Кол-во состояний в СММ},
          ylabel={Время, мс},
          legend pos=north west]
        	\addplot[mark=square,red,thick] table[x=states,y=cudampf] {covid_bench.dat};
        	\addplot[mark=square,blue,thick] table[x=states,y=ns] {covid_median_bench.dat};
        	\addplot[mark=square,green,thick] table[x=states,y=s1] {covid_median_bench.dat};
        	\addplot[mark=square,magenta,thick] table[x=states,y=s2] {covid_median_bench.dat};
        	\addplot[mark=square,black,thick] table[x=states,y=s3] {covid_median_bench.dat};
      	    \legend{CUDAMPF, неспец., ур. 1 спец., ур. 2 спец., ур. 3 спец.}
      	\end{axis}
    \end{tikzpicture}
    \caption{Набор данных из БД \name{PFAM}, меньше --- лучше}
\label{RW_SS}    
\end{figure}

\begin{figure}[h]
\centering
	\begin{tikzpicture}
		\begin{axis}[
	  title={},
          axis x line=bottom,
          axis y line=left,
          xlabel={Кол-во состояний в СММ},
          ylabel={Время, мс},
          legend pos=north west]
        \addplot[mark=square,green,thick] table[x=states,y=st1] {3500_bench.dat};
        \addplot[mark=square,magenta,thick] table[x=states,y=st2] {3500_bench.dat};
        \addplot[mark=square,black,thick] table[x=states,y=st3] {3500_bench.dat};                  				\legend{ур. 1 спец., ур. 2 спец., ур. 3 спец.}
      \end{axis}
    \end{tikzpicture}
    \caption{Время, затраченное на специализацию, меньше --- лучше}
\label{Spec_time_SS}
\end{figure}

Как можно видеть из графиков, с повышением уровня 
специализации обработка набора последовательностей наблюдений 
выполняется быстрее для всех наборов данных.
Стоит также отметить, что специализация третьего уровня либо 
превосходит, либо равна по производительности с реализацией 
из \name{CUDAMPF}.
Время, затраченное на специализацию, возрастает с повышением 
уровня специализации.
Не смотря на то, что для третьего уровня время специализации 
разительно отличается от первого и второго уровня, при 
анализе достаточно большого набора последовательностей это 
время будет незначительным, так как процедура специализации 
делается один раз перед запуском анализа набора.
Из этого можно сделать вывод, что уровень специализации нужно 
выбирать, основываясь на наборе данных.
Например, из таблицы~\ref{runtime} можно сделать вывод, что 
для используемого набора данных самым эффективным является 
второй уровень специализации.
\begin{table}[t!]
  \centering
  \begin{tabular}{||c c c c c c||} 
    \hline
    & CUDAMPF & Initial & 1-level & 2-level & 3-level\\ [0.5ex] 
    \hline\hline
    3 x 3500 & 7907 & 11365 & 8747 & \textbf{5332} & 18858 \\ 
    \hline
    3 x 7000 & \textbf{8862} & 22641 & 17342 & 9630 & 21356 \\
    \hline
    Данные из БД \name{PFAM} & 8347 & 17064 & 12956 & \textbf{7311} & 19581 \\
    \hline
  \end{tabular}
  \caption{Общее время обработки с учетом затрат времени на специализацию, мс}
  \label{runtime}
\end{table}

\achtung{Специализированные реализации с использованием библиотеки \name{CUSP}, показали следующие результаты. ПРОВЕСТИ ЭКСПЕРИМЕНТЫ И ОФОРМИТЬ РЕЗУЛЬТАТЫ}.
\newpage