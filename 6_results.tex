\section{Заключение}
В ходе работы были получены результаты, перечисленные ниже.
\begin{itemize}
	\item Выполнен обзор предметной области:
		\begin{itemize}
			\item рассмотрен алгоритм Витерби и его реализации,такие как HMMER и CUDAMPF;
			\item изучена техника специализации.
		\end{itemize}
	\item Реализованы и протестированы две реализации специализированного алгоритма Витерби, описанного с помощью алгебраической структуры полукольцо “Min-plus” и матричных операций:
		\begin{itemize}
			\item с использованием библиотеки SuiteSparse для выполнения на центральном процессоре;
			\item с использованием библиотеки CUSP для выполнения на графических процессорах общего назначения.
		\end{itemize}
	\item Проведены эксперименты на данных, выложенных в репозитории CUDAMPF. Установлено, что специализированная версия с использованием SuiteSparse в 1,5 раза производительнее CUDAMPF.
\end{itemize}

Статья Viterbi Algorithm Specialization Using Linear Algebra 
была принята на конференции SEIM 2021, а также проведена 
публичная презентация результатов.

Исходя из вышеперечисленных результатов, специализация 
алгоритма Витерби, выраженного методами линейной алгебры, 
скрытой марковской моделью может дать значительный прирост 
производительности.
