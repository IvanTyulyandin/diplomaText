\section{Заключение}
В ходе работы были получены результаты, перечисленные ниже.
\begin{itemize}
	\item Выполнен обзор предметной области:
		\begin{itemize}
			\item рассмотрен алгоритм Витерби и его реализации \name{HMMER} и \name{CUDAMPF};
			\item изучена техника специализации.
		\end{itemize}
	\item Реализованы и протестированы две реализации специализированного алгоритма Витерби, описанного с помощью алгебраической структуры полукольцо \emph{Min-plus} и матричных операций:
		\begin{itemize}
			\item с использованием библиотеки \name{Sui\-te\-Spar\-se:Graph\-BLAS} для выполнения на центральном процессоре;
			\item с использованием библиотеки \name{CUSP} для выполнения на графических процессорах общего назначения.
		\end{itemize}
	\item Проведены эксперименты на данных из репозитория \name{CUDAMPF}. 
Установлено, что специализированная версия с использованием 
\name{Sui\-te\-Spar\-se:Graph\-BLAS} может быть
в несколько раз производительнее \name{CUDAMPF} в зависимости от обрабатываемого набора данных.
На использованных наборах данных было получено ускорение в 1,5-2 раза.
\end{itemize}

Статья \emph{Viterbi Algorithm Specialization Using Linear 
Algebra} была принята на конференции \name{SEIM 2021}, а 
также проведена публичная презентация результатов \achtung{ССЫЛКА?}.

Исходя из вышеперечисленных результатов, специализация 
алгоритма Витерби, выраженного методами линейной алгебры, 
скрытой марковской моделью может дать значительный прирост 
производительности.
