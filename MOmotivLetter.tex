\documentclass[12pt,a4paper]{article}

\hyphenation{op-tical net-works semi-conduc-tor con-tract Et-he-re-um}
\hyphenation{пре-по-да-ва-тель сту-дент вы-чис-ли-тель-но}
\newcommand\name[1]{\texttt{#1}}
\newcommand\achtung[1]{{\color{red}#1}}

\usepackage{indentfirst}
\usepackage{longtable,booktabs,threeparttablex}
\usepackage[export]{adjustbox}
\usepackage{afterpage}
\usepackage[titletoc]{appendix}

\usepackage{fontspec}
% sudo apt install ttf-mscorefonts-installer
% for Times New Roman
\setmainfont{Times New Roman}
\usepackage[nodisplayskipstretch]{setspace}
\onehalfspacing
%\renewcommand{\baselinestretch}{1.5}
\usepackage[russian]{babel}

\usepackage[a4paper]{geometry}
\newgeometry{top=0mm,bottom=20mm,left=25mm,right=20mm,nohead}

\begin{document}
\title{Мотивационное письмо}
\date{}
\author{}
\maketitle
\thispagestyle{empty}
В современном мире все большую роль приобретают информационные технологии.
Объем данных возрастает, в том числе и критически важных, например финансовые транзакции, авторские права и договоры.
Многие учёные думают над тем, чтобы обеспечить сохранность и неизменяемость таких данных.
\emph{Блокчейн} --- это технология распределённого реестра, которая может гарантировать иммутабельность информации в децентрализованной сети.
\emph{Смарт-контракт} --- это программа, описывающая взаимодействие участников блокчейн-сети. 
Мне интересна тема разработки блокчейнов и языков смарт-контрактов.

На базе математико-механического факультета СПбГУ есть лаборатория JetBrains Re\-se\-arch, где активно работает исследовательская группа метавычислений и распределённых технологий под руководством Д.\,А.\,Бе\-ре\-зу\-на.
При поддержке СПбГУ был открыт Центр Технологий Распределённых Реестров (ЦТРР). 
Сотрудники ЦТРР заинтересованы в применении технологии блокчейн, среди них представители различных наук, в том числе и гуманитарных.
Я бы хотел поступить на ма\-те\-ма\-ти\-ко-ме\-ха\-ни\-чес\-кий факультет --- так у меня появится возможность общаться с людьми, которые разносторонне развивают блокчейн, начиная с применения блокчейна в индустрии до создания  формальных языков смарт-контрактов.

Я бакалавр направления <<Программная инженерия>> со средним баллом выше 4,5.
У меня есть публикация в Трудах ИСП РАН, доклад на SYRCoSE, также я дважды выступал на конференции <<Современные технологии в теории и практике программирования>>.
На текущий момент я стажируюсь в <<Linux Foundation>> и развиваю проект, входящий в состав платформы Hyperledger, которая занимается разработкой корпоративных блокчейнов с открытым исходным кодом.

Я считаю, что магистратура по направлению <<Математическое обеспечение и администрирование информационных систем>> на ма\-те\-ма\-ти\-ко-ме\-ха\-ни\-чес\-ком факультете является отличным вариантом продолжения обучения.
Курсы ориентированы на изучение продвинутых технологий разработки программного обеспечения, таких как параллельное программирование и методы машинного обучения.
Также уделяется внимание управлению процессом разработки.
Для меня это важно, так как я хочу стать инженером, который понимает, каким образом создаются современные информационные системы.
На факультете высококлассный преподавательский состав, многие из лекторов и практиков действующие профессионалы в соответствующих областях.

В течение двух лет я хочу повысить свою экспертизу в области блокчейнов, смарт-кон\-трак\-тов и теории формальных языков.
Я предполагаю, что во время обучения в магистратуре я получу качественное образование, как научное, так и техническое.
Это позволит мне работать в R\&D подразделениях крупных компаний и решать проблемы на переднем крае науки.
\end{document}