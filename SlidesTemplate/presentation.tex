\documentclass{beamer}
\usepackage{beamerthemesplit}
\usepackage{wrapfig}
\usetheme{SPbGU}
\usepackage{pdfpages}
\usepackage{amsmath,bm,hyperref,listings,xcolor}
\usepackage{amssymb,amsmath,cancel,cite,color,cmap,float,
	graphicx, multirow,pgfplots,tikz,wrapfig,xcolor}
% weird tex compiler support
\pgfplotsset{compat=1.16}

%\hypersetup{
%    colorlinks = true,
%    anchorcolor = blue,
%}


\usepackage{amsmath}
\usepackage{cmap} 
\usepackage[T2A]{fontenc} 
\usepackage[utf8]{inputenc}
\usepackage[english]{babel}
\usepackage{indentfirst}
\usepackage{amsmath}
\usepackage{multirow}
\usepackage[noend]{algpseudocode}
\usepackage{algorithm}
\usepackage{algorithmicx}
\usepackage{fancyvrb}
\usepackage{tikz}
\usetikzlibrary{shapes,arrows,positioning}
\tikzset{
	node distance=5cm,
	every state/.style={ 
		semithick,
		fill=gray!10
	},
	every edge/.style={
		draw,
		->,>=stealth, 
		auto,
		semithick
	}
}

\newcommand\name[1]{\textsc{#1}}

\def\CC{{C\nolinebreak[4]\hspace{-.05em}\raisebox{.4ex}{\tiny\bf ++}}}

\usepackage[figurename=Рис., tablename=Таблица]{caption}
\setbeamertemplate{caption}[numbered]
\usepackage{graphicx}

% code listings settings
\usepackage{listings}
\usepackage{xcolor}

\definecolor{codegreen}{rgb}{0,0.6,0}
\definecolor{codegray}{rgb}{0.5,0.5,0.5}
\definecolor{codepurple}{rgb}{0.58,0,0.82}
\definecolor{backcolour}{rgb}{0.95,0.95,0.92}
\definecolor{orange}{RGB}{179,36,31}

\lstdefinestyle{mystyle}{
    backgroundcolor=\color{backcolour},   
    commentstyle=\color{codegreen},
    keywordstyle=\color{orange},
    numberstyle=\tiny\color{codegray},
    stringstyle=\color{codepurple},
    basicstyle=\ttfamily\footnotesize,
    breakatwhitespace=false,         
    breaklines=true,                 
    captionpos=b,                    
    keepspaces=true,                 
%    numbers=left,       
    numbersep=5pt,      
    showspaces=false,                
    showstringspaces=false,
    showtabs=false,                  
    tabsize=2
}

\lstdefinelanguage{my_pseudo} {
	morekeywords={function, for, return, let, 
		in, while, if, then, else, elif},
	sensitive=false,
	morecomment=[l]{//},
	morecomment=[s]{/*}{*/},
	morestring=[b]",
	xleftmargin=.2\textwidth, xrightmargin=.2\textwidth
}

\lstset{style=mystyle}
\lstset{language=my_pseudo}
\lstset{
    escapeinside={(*}{*)}
}

% backup settings
\newcommand{\backupbegin}{
   \newcounter{finalframe}
   \setcounter{finalframe}{\value{framenumber}}
}
\newcommand{\backupend}{
   \setcounter{framenumber}{\value{finalframe}}
}


\beamertemplatenavigationsymbolsempty


\title[Специализация алгоритма Витерби]{Исследование применимости специализации алгоритма Витерби скрытой марковской моделью}
\institute[СПбГУ]{
Санкт-Петербургский Государственный Университет\\
Кафедра системного программирования}

% То, что в квадратных скобках, отображается в левом нижнем углу.
\author[Иван Тюляндин]{
\textbf{Автор}: Иван Владимирович Тюляндин, 19.М07-мм
\\
\textbf{Руководитель}: к.ф.-м.н., доцент С.В. Григорьев
\\
\textbf{Консультант}: к.ф.-м.н., ст. преп. СПбГУ Д.А. Березун
\\
\textbf{Рецензент}: ст. преп. СПбПУ М.Х. Ахин}

\date{9 июня 2021}

\begin{document}
{
% Лого университета или организации, отображается в шапке титульного листа
\begin{frame}[plain,noframenumbering]
\begin{center}
  \includegraphics[width=1.7cm]{SPbGU_Logo.png}
\vspace{-3pt}
\hspace{-10pt}
\titlepage
\end{center}

\end{frame}
}


\begin{frame}[fragile]
	\frametitle{Введение}
	\begin{itemize}
		\item Алгоритмы методами линейной алгебры (ЛА) на больших данных
		\vfill
		\item Любые улучшения критичны
			\begin{itemize}
				\vfill
				\item оборудование
				\vfill
				\item изменение алгоритма
			\end{itemize}
		\vfill	
		\item Часть данных может быть зафиксирована
		\vfill	
		\item Применение специализации
	\end{itemize}
\end{frame}


\begin{frame}[fragile]
	\frametitle{Специализация}
	Два типа параметров
	\begin{itemize}
		\item статические --- т.е. зафиксированные
		\item динамические --- все остальные
	\end{itemize}
	\vfill
	\begin{block}{Специализация}
		Техника преобразования программ для оптимизации использования статических данных с целью уменьшить количество вычислений
	\end{block}
	\vfill
	Применение специализации к алгоритмам, выраженными методами ЛА, не изучено
\end{frame}

\begin{frame}[fragile]
	\frametitle{Алгоритм Витерби}
	\begin{itemize}
		\item применяется во многих областях
		\vfill
		\item выражается методами ЛА
		\vfill
		\item два параметра: скрытая марковская модель (СММ) и последовательность наблюдений
		\vfill
		\item на практике СММ зафиксирована, меняется только последовательность
	\end{itemize}
\end{frame}


\begin{frame}[fragile]
	\frametitle{Цель и задачи}
	\begin{block}{Цель работы}
		Исследовать применимость специализации к алгоритму Витерби, который описан методами ЛА,
при условии, что СММ является статическим параметром
	\end{block}
	\vfill
\begin{itemize}
	\item сделать обзор предметной области
		\begin{itemize}
			\item рассмотреть алгоритм Витерби и его существующие реализации
			\item описать технику специализации
		\end{itemize}
	\item разработать специализированный алгоритм Витерби, реализовать и протестировать корректность реализации
	\item провести эксперименты по сравнению производительности специализированного алгоритма с неспециализированной версией и существующими реализациями
\end{itemize}
\end{frame}

\begin{frame}[fragile]
	\frametitle{Скрытая марковская модель}
	\begin{block}{Скрытая марковская модель}
		Вероятностный детерминированный автомат, каждое состояние которого создает наблюдение
		\begin{itemize}
			\item $\mathit{S_{1..N}}$ --- $N$ состояний
			\item $\mathit{O_{1..K}}$ --- $K$ возможных наблюдений
			\item $\mathit{B_{1..N}}$ --- вероятности для состояний $\mathit{S_{1..N}}$ быть стартовыми 
			\item $\mathit{T_{1..N, 1..N}}$ --- матрица переходов, $\mathit{T_{i,j}}$ --- это вероятность перехода из $\mathit{S_{i}}$ в $\mathit{S_{j}}$
			\item $\mathit{E_{1..N, 1..K}}$ --- матрица наблюдений, $\mathit{E_{i,j}}$ --- это вероятность создать наблюдение 
$\mathit{O_{j}}$ в состоянии $\mathit{S_{i}}$
		\end{itemize}
	\end{block}
	\vfill
\end{frame}


\begin{frame}[fragile]
	\frametitle{Алгоритм Витерби}
СММ и последовательность наблюдений
\vfill
Алгоритм Витерби, выраженный с помощью полукольца \emph{Min-plus}
\vfill
Для всех вероятностей \emph{p} из СММ:
\[  t(p) =
    \begin{cases}
      p > 0: & -1 * log_{2}(p))\\
      p = 0: & +\infty
    \end{cases}       
\]
\end{frame}


\begin{frame}[fragile]
	\frametitle{Алгоритм Витерби методами ЛА}
	Для всех наблюдений $o$:
\[
  P(o) =
  \begin{pmatrix}
    t(E[1,o]) & \hdots & +\infty \\
    \vdots & \ddots & \vdots\\
    +\infty & \hdots & t(E[N,o]) \\
  \end{pmatrix}
\]
	\vfill
	\begin{block}{Алгоритм Витерби методами ЛА}
		Обработка первого наблюдения из последовательности $Obs$:
		\[\mathit{Probs}_{1} = P(\mathit{Obs[1]}) \times B\]
		Оставшаяся часть $Obs$:
	\[\mathit{Probs}_{t} = P(\mathit{Obs[t]}) \times T^{\top} \times \mathit{Probs}_{t - 1}\]	
	\end{block}
\end{frame}


\begin{frame}[fragile]
	\frametitle{Алгоритм специализации}
	Как использовать данные СММ?
	\vfill
	Для всех наблюдений $o$ можно вычислить: 
	\[P(\mathit{o}) \times B \text{, далее как } PB(o)\]
	\[P(\mathit{o}) \times T^{\top} \text{, далее как } PT(o)\] 
	\vfill
	\begin{block}{Алгоритм Витерби первого уровня специализации}
		\[\mathit{Probs}_{1} = PB(Obs[1])\]
	
		\[\mathit{Probs}_{t} = PT(Obs[t]) \times \mathit{Probs}_{t - 1}\]

	\end{block}
\end{frame}


\begin{frame}[fragile]
	\frametitle{Алгоритм специализации}
	Матричное умножение ассоциативно
	\vfill
	Обработка $o_1$ и $o_2$, если столбец $Probs_0$ известен?
\begin{align}
  \mathit{Probs}_{2} &= \mathit{PT}(\mathit{o}_{2}) \times \mathit{Probs}_{1}\nonumber\\
  &= \mathit{PT}(\mathit{o}_{2}) \times (\mathit{PT}(\mathit{o}_{1}) \times \mathit{Probs}_{0}) \nonumber\\
  & =(\mathit{PT}(\mathit{o}_{2}) \times \mathit{PT}(\mathit{o}_{1})) \times \mathit{Probs}_{0}
\end{align}
\vfill
Можно вычислить
$\mathit{PT}(\mathit{o}_{2}) \times \mathit{PT}(\mathit{o}_{1})$. 
Два наблюдения одним умножением, т.е. второй уровень!
\vfill
Этот подход можно применить для повышения уровня
\end{frame}


\begin{frame}[fragile]
	\frametitle{Анализ операций}
	Пусть длина $\mathit{Obs}$ обозначается как $lo$, количество состояний СММ как $N$, количество возможных наблюдений СММ как $K$
	\vfill
	\begin{block}{Неспециализированная версия}
		Матричных умножений: \[1 + 2 \ast (\mathit{lo} - 1)\]
	\end{block}
	\vfill
	\begin{block}{Специализированная версия уровня $M$}
		Матричных умножений: \[\mathit{(lo - 1) / M + (lo - 1)\ mod\ M}\]
		Дополнительной памяти: $\mathit{K^{M}}$ матриц, каждая $N \times N$ 
	\end{block}
	
\end{frame}


\begin{frame}[fragile]
	\frametitle{Реализация}
Нужно и для CPU, и для GPGPU
	\vfill
	Библиотеки \name{SuiteSparse:GraphBLAS} и \name{cuASR}
	\vfill
	Проведено тестирование
	\begin{itemize}
		\item на корректность реализации алгоритма Витерби
		\item на сохранение семантики
		\item отсутствие утечек памяти
	\end{itemize}
\end{frame}


\begin{frame}[fragile]
	\frametitle{Эксперименты}
	\begin{itemize}	
		\item Ubuntu 20.04 \newline Intel Core i7-6700 3.40 
ГГц, 64 Гб RAM \newline NVIDIA GeForce GTX 1070, 8Гб, 1920 CUDA-ядер
		\vfill
		\item Сравнение специализированной версии c неспециализированной и \name{CUDAMPF}
		\vfill
		\item 24 СММ из репозитория \name{CUDAMPF}, 4 набора данных
		\vfill 
		\item Медиана из 10 запусков
	\end{itemize}
\end{frame}


\begin{frame}[fragile]
	\frametitle{Эксперименты}
\begin{figure}[h!]
\centering
	\begin{tikzpicture}
	\begin{axis}[
	      title={},
          axis x line=bottom,
          axis y line=left,
          xlabel={Кол-во состояний в СММ},
          ylabel={Время, мс},
          legend pos=outer north east]
        \addplot table [x=States, y=CUDAMPF, col sep=comma] {../covid-19.csv};
		\addplot table [x=States, y=GraphBLAS, col sep=comma] {../covid-19.csv};
		\addplot[color=green,mark=diamond*] table [x=States, y=GraphBLAS_spec_1, col sep=comma] {../covid-19.csv};
       	\legend{CUDAMPF, неспец., ур. 1 спец.}
	\end{axis}
    \end{tikzpicture}
    \caption{GraphBLAS, набор данных из БД \name{PFAM}, меньше --- лучше}
\label{RW_SS}    
\end{figure}
\end{frame}


\begin{frame}[fragile]
	\frametitle{Эксперименты}
\begin{figure}[h!]
\centering
	\begin{tikzpicture}
	\begin{axis}[
	      title={},
          axis x line=bottom,
          axis y line=left,
          xlabel={Кол-во состояний в СММ},
          ylabel={Время, мс},
          legend pos=outer north east]
		\addplot table [x=States, y=GraphBLAS_spec_1_prep, col sep=comma] {../50_3500.csv};
		\addplot table [x=States, y=GraphBLAS_spec_2_prep, col sep=comma] {../50_3500.csv};
       	\legend{ур. 1 спец., ур. 2 спец.}
	\end{axis}
    \end{tikzpicture}
    \caption{GraphBLAS, время на специализацию, меньше --- лучше}
\label{Spec_time_SS}
\end{figure}
\end{frame}

\begin{frame}[fragile]
	\frametitle{Эксперименты}
\begin{figure}[h!]
\centering
	\begin{tikzpicture}
	\begin{axis}[
	      title={},
          axis x line=bottom,
          axis y line=left,
          xlabel={Кол-во состояний в СММ},
          ylabel={Время, мс},
          legend pos=outer north east]
        \addplot table [x=States, y=CUDAMPF, col sep=comma] {../covid-19.csv};
		\addplot table [x=States, y=cuASR, col sep=comma] {../covid-19.csv};
		\addplot[color=green,mark=diamond*] table [x=States, y=cuASR_spec_1, col sep=comma] {../covid-19.csv};
		\addplot table [x=States, y=cuASR_spec_2, col sep=comma] {../cuASR_covid-19.csv};
       	\legend{CUDAMPF, неспец., ур. 1 спец., ур. 2 спец.}
	\end{axis}

    \end{tikzpicture}
    \caption{cuASR, набор данных из БД \name{PFAM}, меньше --- лучше}
\label{RW_CUSP}    
\end{figure}
\end{frame}

\begin{frame}[fragile]
	\frametitle{Эксперименты}
\begin{figure}[h!]
\centering
	\begin{tikzpicture}
	\begin{axis}[
	      title={},
          axis x line=bottom,
          axis y line=left,
          xlabel={Кол-во состояний в СММ},
          ylabel={Время, мс},
          legend pos=outer north east]
		\addplot table [x=States, y=cuASR_spec_1_prep, col sep=comma] {../50_3500.csv};
		\addplot table [x=States, y=cuASR_spec_2_prep, col sep=comma] {../cuASR_50_3500.csv};
       	\legend{ур. 1 спец., ур. 2 спец.}
	\end{axis}
    \end{tikzpicture}
    \caption{cuASR, время на специализацию, меньше --- лучше}
\label{Spec_time_CUSP}
\end{figure}
\end{frame}

\begin{frame}[fragile]
	\frametitle{Результаты}
\begin{itemize}
	\item Выполнен обзор предметной области
	\item Реализованы и протестированы две реализации специализированного алгоритма Витерби
		\begin{itemize}
			\item \name{Sui\-te\-Spar\-se:Graph\-BLAS} для выполнения на CPU
			\item \name{cuASR} для выполнения на GPGPU
		\end{itemize}
	\item Проведены эксперименты на данных из репозитория \name{CUDAMPF}
\vfill
Специализированная версия алгоритма Витерби превосходит неспециализированную
\vfill
\name{SEIM 2021}: статья \emph{Viterbi Algorithm Specialization Using Linear Algebra}
\end{itemize}
\end{frame}


% -------------BACKUP-SLIDES--------

\appendix
\backupbegin

\begin{frame}
	\frametitle{Ответы на вопросы рецензента}
\begin{enumerate}
	\item Подход не зависит от бионформатики и применим в областях, где СММ может быть фиксирована
	\vfill
	\item Не было найдено инструментов для специализации с возможностью оптимизации ЛА
	\vfill
	\item На графиках указано время обработки без учёта специализации
	\vfill
	\item Точная причина не установлена
	\vfill
	\item Количество CUDA-ядер у NVIDIA GTX 1070 равно 1920 
	\vfill
	\item Указано время для обработки СММ с кол-вом состояний менее 2000
\end{enumerate}
\end{frame}

\begin{frame}[fragile]
	\frametitle{Пример специализации: возведение в степень}
\begin{lstlisting}[escapeinside={(*}{*)}]
function f(x, n)
  if n == 0 then 1
  elif even(x) then f(x, n/2)^2
  else x * f(x, (n-1)/2)^2
\end{lstlisting}
	\vfill
	Предположим $\mathit{n = 5}$, т.е. это статический параметр
	\vfill
\begin{lstlisting}[escapeinside={(*}{*)}]
function f_spec(x) = x * (x^2)^2
\end{lstlisting}
	\vfill
	Специализация бесполезна, если $x$ --- статический параметр
\end{frame}


\begin{frame}[fragile]
	\frametitle{СММ: пример}
	\centering
\begin{figure}
	\begin{tikzpicture}
		\node (H) {
			\begin{tabular}{|c|c|}
				\hline
				\multicolumn{2}{|c|} H \\
				\hline
				A & 0.2 \\
				C & 0.3 \\
				T & 0.3 \\
				G & 0.2 \\
				\hline
			\end{tabular}
		};
		\node[right of=H] (L) {
			\begin{tabular}{|c|c|}
				\hline
				\multicolumn{2}{|c|} L \\
				\hline
				A & 0.3 \\
				C & 0.2 \\
				T & 0.2 \\
				G & 0.3 \\
				\hline
			\end{tabular}
		};
		\draw (H) edge[loop above] node {\tt 0.5} (H);
		\draw (H) edge[bend left] node {\tt 0.5} (L);
		\draw (L) edge[loop above] node {\tt 0.6} (L);
		\draw (L) edge[bend left] node {\tt 0.4} (H);
	\end{tikzpicture}
	\label{fig:HMM}
\end{figure}
\begin{figure}
	\begin{tabular}{c}
		Вероятности быть стартовым состоянием\\
		\begin{tabular}{|c|c|}
			\hline
			H & L \\
			\hline
			0.5 & 0.5 \\
			\hline
		\end{tabular} \\
	\end{tabular}
\end{figure}
\end{frame}

\begin{frame}[fragile]
	\frametitle{Полукольцо \emph{Min-plus}}
Переопределяем '+' как \emph{min}, '*' как  \emph{plus}, $\infty$ и $0$ как нейтральные элементы
	\vfill
	Пример матричного умножения
\[
  \begin{pmatrix}
    0 & 1 \\
    +\infty & 2
  \end{pmatrix}
  \times
  \begin{pmatrix}
    3 \\
    4
  \end{pmatrix}
  =
  \begin{pmatrix}
    min(0 + 3, 1 + 4) \\
    min(+\infty + 3, 2 + 4)
  \end{pmatrix}
  =
  \begin{pmatrix}
    3 \\
    6
  \end{pmatrix}
\]
\end{frame}


\begin{frame}
	\frametitle{О повышении уровня}
	Обработка $o_{t-2}$, $o_{t-1}$ и $o_t$, если столбец $Probs_{t-3}$ известен?
	\vfill
\begin{align}
  \mathit{Probs}_{t} = \mathit{PT}(o_t) \times \mathit{PT}(o_{t-1}) \times \mathit{PT}(o_{t-2}) \times \mathit{Probs}_{t - 3} 
\label{lvl_3}
\end{align}
	\vfill
	Все произведения $PT(o)$ могут быть вычислены по данным из СММ
\end{frame}

\begin{frame}
	\frametitle{СММ в БД PFAM}
\begin{figure}[h!]
  \centering
  \includegraphics[width=0.80\columnwidth]{../mean_HMM.png}
  \caption{График зависимости количества СММ от количества состояний в них}
  \label{HMM_mean}
\end{figure}
\end{frame}

\begin{frame}[fragile]
	\frametitle{Эксперименты}
\begin{table}[h!]
  \centering
  \begin{tabular}{||c c c c c||} 
    \hline
    & CUDAMPF & неспец. & Ур. 1 & Ур. 2 \\ [0.5ex] 
    \hline\hline
    3 x 3500 & 4854 & 10765 & 8062 & 215329 \\ 
    \hline
    3 x 7000 & 9209 & 21062 & 16152 & 387464 \\
    \hline
    Набор из \name{PFAM} & 8796 & 15864 & 12036 & 298269 \\
    \hline
    50 x 3500 & 103036 & 176263 & 134259 & 2921104 \\
    \hline
  \end{tabular}
  \caption{GraphBLAS, общее время обработки с учетом затрат времени на специализацию, мс}
  \label{runtime}
\end{table}
\end{frame}

\begin{frame}[fragile]
	\frametitle{Эксперименты}
\begin{table}[h!]
  \centering
  \begin{tabular}{||c c c c c||} 
    \hline
    & CUDAMPF & неспец. & Ур. 1 & Ур. 2\\ [0.5ex] 
    \hline\hline
    3 x 3500 & 3666 & 154415 & 77758 & 85977 \\ 
    \hline
    3 x 7000 & 7053 & 305578 & 153216 & 145176 \\
    \hline
    Набор из \name{PFAM} & 6777 & 241687 & 114214 & 114766 \\
    \hline
    50 x 3500 & 78930 & 2596532 & 1271980 & 1025473 \\
    \hline
  \end{tabular}
  \caption{cuASR, общее время обработки с учетом затрат времени на специализацию, кол-во состояний СММ меньше 2000, мс}
  \label{runtime_CUSP}
\end{table}
\end{frame}

\backupend

\end{document}
