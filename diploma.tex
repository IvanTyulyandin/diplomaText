% Тут используется класс, установленный на сервере Papeeria. На случай, если
% текст понадобится редактировать где-то в другом месте, рядом лежит файл matmex-diploma-custom.cls
% который в момент своего создания был идентичен классу, установленному на сервере.
% Для того, чтобы им воспользоваться, замените matmex-diploma на matmex-diploma-custom
% Если вы работаете исключительно в Papeeria то мы настоятельно рекомендуем пользоваться
% классом matmex-diploma, поскольку он будет автоматически обновляться по мере внесения корректив
%

% По умолчанию используется шрифт 14 размера. Если нужен 12-й шрифт, уберите опцию [14pt]
%\documentclass[14pt]{matmex-diploma}
\documentclass[14pt]{matmex-diploma-custom}

\hyphenation{op-tical net-works semi-conduc-tor con-tract Et-he-re-um}
\hyphenation{пре-по-да-ва-тель сту-дент вы-чис-ли-тель-но}
\newcommand\name[1]{\textsc{#1}}
\newcommand\achtung[1]{{\color{red}#1}}

\usepackage{longtable,booktabs,threeparttablex}
%\usepackage{makecell}
\usepackage[export]{adjustbox}
\usepackage{afterpage}


\begin{document}
% Год, город, название университета и факультета предопределены,
% но можно и поменять.
% Если англоязычная титульная страница не нужна, то ее можно просто удалить.
\filltitle{ru}{
    chair              = {Программная инженерия\\ Кафедра системного программирования},
    title              = {Специализация алгоритма Витерби на GPU скрытой марковской моделью},
    % Здесь указывается тип работы. Возможные значения:
    %   coursework - Курсовая работа
    %   diploma - Диплом специалиста
    %   master - Диплом магистра
    %   bachelor - Диплом бакалавра
    type               = {master},
    author             = {Тюляндин Иван Владимирович},
    supervisorPosition = {к.\,ф.-м.\,н., доцент},
    supervisor         = {С.\,В. Григорьев},
    consultantPosition   = {программист ООО "Интеллиджей Лабс"\\ к.\,ф.-м.\,н.},
    consultant           = {Д.\,А. Березун},
    reviewerPosition   = {},
    reviewer           = {}
%   university         = {Санкт-Петербургский Государственный Университет},
%   faculty            = {Математико-механический факультет},
%   city               = {Санкт-Петербург},
%   year               = {2013}
}
\filltitle{en}{
    chair              = {Software Engineering},
    title              = {Specialization of GPU implemented Viterbi algorithm with hidden Markov model},
    author             = {Ivan Tyulyandin},
    supervisorPosition = {Assistant Professor, PhD},
    supervisor         = {Semyon Grigorev},
    consultantPosition   = {IntelliJ Labs Co. Ltd developer\\ PhD},
    consultant           = {Daniil Berezun},
    reviewerPosition   = {},
    reviewer           = {}
    % chairHeadPosition  = {professor},
    % chairHead          = {Christobal Junta},
}
\maketitle
\tableofcontents
% У введения нет номера главы

\section*{Введение}

Скрытые марковские модели~\cite{HMM_wiki}.

\section{Постановка задачи}
Цель данной работы --- исследовать применимость специализации алгоритма Витерби, реализованного на GPGPU.

Были поставлены следующие задачи:
\begin{itemize}
	\item сделать обзор предметной области и существующих решений задачи 
		гомологичности;
	\item написать специализатор алгоритма Витерби на СММ;
	\item провести сравнительный анализ специализированной программы с
		существующими решениями.
\end{itemize}


\setmonofont[Mapping=tex-text]{CMU Typewriter Text}
\bibliographystyle{ugost2008ls}
\bibliography{diploma.bib}
\end{document}
