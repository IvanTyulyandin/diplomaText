% Тут используется класс, установленный на сервере Papeeria. На случай, если
% текст понадобится редактировать где-то в другом месте, рядом лежит файл matmex-diploma-custom.cls
% который в момент своего создания был идентичен классу, установленному на сервере.
% Для того, чтобы им воспользоваться, замените matmex-diploma на matmex-diploma-custom
% Если вы работаете исключительно в Papeeria то мы настоятельно рекомендуем пользоваться
% классом matmex-diploma, поскольку он будет автоматически обновляться по мере внесения корректив
%

% По умолчанию используется шрифт 14 размера. Если нужен 12-й шрифт, уберите опцию [14pt]
%\documentclass[14pt]{matmex-diploma}
\documentclass[14pt]{matmex-diploma-custom}

\hyphenation{op-tical net-works semi-conduc-tor con-tract Et-he-re-um}
\hyphenation{пре-по-да-ва-тель сту-дент вы-чис-ли-тель-но}
\newcommand\name[1]{\textsc{#1}}
\newcommand\achtung[1]{{\color{red}#1}}

\usepackage{longtable,booktabs,threeparttablex}
%\usepackage{makecell}
\usepackage[export]{adjustbox}
\usepackage{afterpage}


\begin{document}
% Год, город, название университета и факультета предопределены,
% но можно и поменять.
% Если англоязычная титульная страница не нужна, то ее можно просто удалить.
\filltitle{ru}{
    chair              = {Программная инженерия\\ Кафедра системного программирования},
    title              = {Проектирование и реализация среды исполнения смарт-контрактов для блокчейна Hyperledger Iroha},
    % Здесь указывается тип работы. Возможные значения:
    %   coursework - Курсовая работа
    %   diploma - Диплом специалиста
    %   master - Диплом магистра
    %   bachelor - Диплом бакалавра
    type               = {bachelor},
    author             = {Тюляндин Иван Владимирович},
    supervisorPosition = {ст. преп.},
    supervisor         = {Кириленко Я. А.},
    consultantPosition   = {программист ООО "Интеллиджей Лабс"\\ к.\,ф.-м.\,н.},
    consultant           = {Березун Д. А.},
    reviewerPosition   = {Генеральный директор ООО «Сорамитсу Лабс»\\},
    reviewer           = {Салахиев К. Р.}
%   university         = {Санкт-Петербургский Государственный Университет},
%   faculty            = {Математико-механический факультет},
%   city               = {Санкт-Петербург},
%   year               = {2013}
}
\filltitle{en}{
    chair              = {Software Engineering},
    title              = {Design and implementation of smart contracts execution environment for Hyperledger Iroha blockchain},
    author             = {Ivan Tyulyandin},
    supervisorPosition = {Senior lecturer},
    supervisor         = {Iakov Kirilenko},
    consultantPosition   = {IntelliJ Labs Co. Ltd developer\\ PhD},
    consultant           = {Daniil Berezun},
    reviewerPosition     = {Soramitsu Labs CEO},
    reviewer             = {Kamil Salakhiev}
    % chairHeadPosition  = {professor},
    % chairHead          = {Christobal Junta},
}
\maketitle
\tableofcontents
% У введения нет номера главы


\section*{Введение}

В бионформатике необходимо обрабатывать большие объемы данных.
Одна из задач --- проверка гомологичности протеина семейству протеинов, 
то есть определение сходства.
Протеин состоит из произвольной по длине комбинации 20 стандартных аминокислот.
В биологии каждая аминокислота кодируется в виде символа.
Используя эту кодировку, можно получить символьную последовательность,
которая будет полностью описывать протеин.
Задача осложняется тем, что протеины, даже имея различную последовательность,
могут принадлежать к одному и тому же семейству.
Такое возможно из-за физико-биологических свойств стандартных аминокислот.

На основе известных данных о гомологичности группы протеинов 
можно построить \emph{профиль семейства} --- вероятностную модель, 
которая описывает семейство протеинов с помощью 
скрытой марковской модели (далее СММ) со специальной структурой,
учитывающей возможность вставки, замены и удаления аминокислот~\cite{HMM_Eddy}.
Для работы с СММ используется алгоритм Витерби, впервые предложенный в статье~\cite{Viterbi}.

С экспоненциальным ростом количества известных протеинов возрастает
необходимость ускорения их обработки.
Для решения задачи гомологичности используются различные варианты оптимизации,
начиная от вероятностных алгоритмов~\cite{MSV_Eddy} 
до использования графических процессоров общего назначения (GPGPU)~\cite{cudampf}.

В область зрения исследователей не попали методы, 
основанные на метавычислениях.
Иногда приходится работать с одним и тем же семейством протеинов.
Можно использовать этот факт для создания версии алгоритма Витерби, 
которая будет определять гомологичность только с конкретным семейством протеинов.
В теории метавычислений есть техника \emph{специализации}~\cite{Jones_spec}.
Параметры, подаваемые на вход алгоритму, можно условно разделить на две группы:
статические и динамические.
Статические параметры не меняются от запуска к запуску, 
а динамические могут принимать множество различных значений.
Суть специализации в подстановке известных статических 
данных в алгоритм и упрощении выражений, 
в которых все параметры известны после подстановки.
Ожидается, что специализированная программа будет работать быстрее за счет того, что часть вычислений, связанных с известными данными, будет выполнена на стадии генерации специализированной программы,
и их результат будет сохранен в коде этой программы.
Программа, генерирующая специализированную программу, называется \emph{специализатор}.

Для решения задачи гомологичности протеина с конкретным семейством протеинов 
можно сгенерировать специализированную версию алгоритма Витерби,
где СММ, описывающая семейство, будет статическим параметром, 
а протеин -- динамическим.
Возможно, что такая версия окажется оптимальнее по времени работы 
по сравнению с обобщенным алгоритмом Витерби.

\section{Обзор}
В этой главе будут рассмотрены среды исполнения и языки смарт-контрактов, а так же некоторые особенности \name{Hyperledger Iroha}.

\subsection{Языки и среды исполнения смарт-контрактов}
В 1997 году Ник Сабо (Nick Szabo) предложил концепцию смарт-контрактов~\cite{Szabo_SC}.
Смарт-контракт --- это программа, которая описывает взаимодействие участников блокчейн-сети. 
При сравнении с традиционным бумажным контрактом, смарт-контракт имеет однозначную семантику и выполняется автоматически при достижении определенных условий.
Результат выполнения смарт-контракта легко подтверждается, так как он будет записан в историю транзакций блокчейна.
На сегодняшний день существует множество различных языков смарт-контрактов и блокчейнов, которые могут исполнять программы на этих языках.

Смарт-контракт всегда должен завершаться для продолжения работы блокчейн-сети.
Если язык смарт-контрактов Тьюринг-полный, то среда исполнения должна предоставлять механизм, который будет ограничивать тем или иным способом количество операций смарт-контракта.
В случае \name{Ethereum} каждая инструкция стоит определенное количество \emph{газа}, цена которого выражена во внутренней криптовалюте \name{Et\-he\-re\-um}.
В \name{Hyperledger Fabric} имеется ограничение на время выполнения кода смарт-контракта.
Для языка \name{Rholang}~\cite{Rholang}, основанном на \name{Rho-calculus}~\cite{RhoCalculus}, выставляется лимит по количеству применений правил редукции.

Далее будут рассмотрены несколько сред исполнения смарт-кон\-трак\-тов и языки, которые данные среды поддерживают.

\subsubsection{Языки смарт-контрактов}
В этом параграфе рассмотрены языки смарт-контрактов с учетом их парадигм и свойств, таких как Тьюринг-полнота, механизм ограничения выполнения смарт-контракта платформой, на которой смарт-контракт исполняется, и системы типов.

На данный момент существуют языки смарт-контрактов с различным уровнем абстракции. 
\emph{Низкоуровневые языки} (low-level) предназначены для непосредственного выполнения средой исполнения.
Многие концепции, такие как семантика, вычислительная модель, система ограничения выполнения и типизация часто описываются на этом уровне.
Примеры таких языков --- \name{EVM}~\cite{EthereumYellowPaper}, \name{Bitcoin Script}~\cite{BitcoinScript} и \name{Michelson}~\cite{Michelson}.
\emph{Высокоуровневые языки} (high-level), такие как \name{Solidity}~\cite{Solidity}, \name{Flint}~\cite{Flint} и \name{Liquidity}~\cite{liquidity},
упрощают процесс разработки смарт-контрактов за счет повышенной читаемости, наличия более абстрактных синтаксических конструкций и системы типов.
\emph{Промежуточные языки} (intermediate-level) смарт-контрактов являются своего рода компромиссом между высокоуровневыми и низкоуровневыми языками по степени абстракции.
Как правило, они спроектированы для упрощения формальной верификации или статического анализа исходного кода, с учетом вычислительной модели, системы типов, семантики и других формализмов.
\name{Scilla}~\cite{Scilla} является промежуточным языком смарт-контрактов.

В ходе обзора языков смарт-контрактов на конференцию SYRCoSE 2019 в соавторстве была написана
обзорная статья \emph{A Survey of Smart Contract Safety and Pro\-gram\-ming Languages}, которая принята к публикации в сборнике трудов ИСП РАН.
В приложении~\ref{appendixA} приведена сводная таблица по языкам смарт-кон\-трак\-тов и их свойствам из данной статьи.
В ней приведены следующие характеристики языков смарт-контрактов: название, уровень абстракции, текущее состояние разработки, проект, для которого язык предназначен, парадигма, способ ограничения выполнения смарт-контракта целевой платформой и Тьюринг-полнота соответственно.

Было выявлено, что \name{Ethereum} является наиболее популярной платформой для работы со смарт-контрактами.
Экосистема данного блокчейна развита: существует множество языков с различными подходами, сред разработки и статических анализаторов.
Аналогичной экосистемы нет ни у одного из всех рассмотренных блокчейнов.


\subsubsection{\name{Ethereum Virtual Machine}}
% https://habr.com/ru/post/340928/
\name{Ethereum Virtual Machine} (сокращенно \name{EVM}) --- Тьюринг-полная виртуальная стековая машина блокчейна \name{Ethereum}.
Смарт-контракты для этой платформы написаны на байткоде~\cite{EthereumYellowPaper}.
Под эту среду исполнения смарт-контрактов существует множество языков ~\cite{Bamboo, Flint, IELE, Logikon, Solidity, SolidityX, Vyper, LLL, Yul}, которые компилируются в байткод \name{EVM}.

Есть два участка памяти, куда \name{EVM} может записывать значения во время выполнения кода смарт-контракта --- \emph{memory} и \emph{storage}.
Memory является временным хранилищем данных, необходимым для записи промежуточных значений.
Размер машинного слова \name{EVM} 256 битов.
Можно провести аналогию, что memory для \name{EVM} --- это как оперативная память для компьютера.
Ячейки memory адресуются от 0 до $2^{256} - 1$ и содержат байт информации.
Storage представляет из себя хранилище пар вида ключ-значение, которые описывают текущее состояние переменных смарт-контракта.
В отличие от memory, данные storage записываются в блокчейн.
Размер storage равен $2^{256}$ ячеек, каждая из них хранит машинное слово.

Аккаунты в блокчейн сети \name{Ethereum} бывают двух видов: \emph{пользовательские} и \emph{контракты}.
Аккаунты-контракты содержат код, который может быть вызван пользовательским аккаунтом или кодом другого контракта.

Существует множество реализаций клиентов для работы с распределённой блокчейн сетью \name{Ethereum}, например \name{Geth}\footnote{https://geth.ethereum.org/} и \name{Aleth}\footnote{http://www.ethdocs.org/en/latest/ethereum-clients/cpp-ethereum/}.
Также есть проекты, которые используют только виртуальную машину \name{Ethereum} и её байт-код, таким проектом является \name{Hy\-per\-led\-ger Bur\-row}~\cite{HLBurrow}.

\subsubsection{Выбор среды исполнения}
Для интеграции в \name{Hyperledger Iroha} была выбрана среда исполнения смарт-кон\-трак\-тов из проекта \name{Hy\-per\-led\-ger Bur\-row}.	
Этот проект реализует приватный блокчейн с возможностью выполнения смарт-кон\-трак\-тов и написан на языке \name{Go}.
Ключевыми особенностями реализации являются алгоритм консенсуса \name{Ten\-der\-mint}~\cite{Tendermint}, наличие программного интерфейса для удаленного вызова процедур, возможность выставлять права доступа к данным и на выполнение операций внутри сети, а также виртуальная машина для смарт-контрактов.

Эта среда исполнения обладает следующими преимуществами: 1) выполнение смарт-контрактов, написанных на байт-коде \name{EVM}; 2) наличие программного интерфейса для взаимодействия; 3) проект поддерживается \name{Hy\-per\-led\-ger}; 4) есть примеры интеграции с проектами \name{Hy\-per\-led\-ger Fabric}~\cite{HLFabricEVM} и \name{Hyperledger Sawtooth}~\cite{HLSeth}.
Так как \name{Ethereum} де-факто является самой популярной платформой для работы со смарт-контрактами, программистам будет проще адаптироваться к разработке на \name{Hy\-per\-led\-ger Iro\-ha}.
Важную роль играет принадлежность к \name{Hy\-per\-led\-ger} --- при возникновении проблем и вопросов на этапе интеграции виртуальной машины можно обратиться к сообществу разработчиков указанных ранее проектов.

У виртуальной машины внутри есть доступ к участку памяти me\-mo\-ry, а для реализации операций над участком памяти storage необходимо обращаться к истории блокчейна.
Программный интерфейс для взаимодействия описывает все методы, которые нужны виртуальной машине, чтобы получать, добавлять и обновлять данные блокчейна.


\subsection{\name{Hyperledger Iroha}}
\name{Hyperledger Iroha}~\cite{iroha} --- это приватный блокчейн, обладающий функциональностью для управления активами и сущностями, вводимые пользователями.
В нем используется алгоритм консенсуса \name{YAC}~\cite{YAC}, который основан на решении задачи о византийских генералах.
Данный фреймворк для распределенных хранилищ ориентирован на использование на мобильных устройствах.
Проект написан на языке \name{C++} с использованием библиотек \name{Boost}\footnote{https://www.boost.org/}, \name{Protobuf}\footnote{https://developers.google.com/protocol-buffers/}, \name{GTest}\footnote{https://github.com/google/googletest} и множества других.
В качестве хранилища истории блокчейна сети используется локальная для каждого участника база данных \name{PostgreSQL}\footnote{https://www.postgresql.org/}.

Как и во многих блокчейн сетях, пользователи \name{Hyperledger Iroha} выполняют различные действия с помощью транзакций, которые формируются из набора специальных команд.
Используя их, участники могут пересылать активы и управлять правами доступа, например, создать новую сущность или передать право владением активом другому участнику.

Прежде чем принять транзакцию, всем участникам сети нужно подтвердить её корректность.
Есть два этапа валидации --- \emph{stateless} и \emph{sta\-te\-ful}.
Команды должны быть сформированы согласно определённым правилам.
Во время stateless валидации контролируется соответствие этим правилам.
Далее проводится stateful валидация. 
На этом этапе проверяется выполнимость отправленной транзакции, например, права доступа или наличие активов.

\section{Взаимодействие среды исполнения смарт-кон\-трак\-тов с \name{Hy\-per\-led\-ger Iro\-ha}}
В этой главе будет рассмотрена схема взаимодействия среды исполнения смарт-контрактов и программный интерфейс, который эту схему реализует.

\subsection{Схема взаимодействия}
\label{Interaction}

\begin{figure}[b!]
  \centering
  \includegraphics[width=0.6\columnwidth]{interaction.png}
  \caption{Диаграмма последовательности взаимодействия}
  \label{interaction}
\end{figure}

Для создания инфраструктуры для среды исполнения смарт-кон\-трак\-тов необходимо разработать схему взаимодействия этой среды и блокчейна \name{Hy\-per\-led\-ger Iro\-ha}.
Среда исполнения должна уметь получать и записывать данные в блокчейн.

На рисунке~\ref{interaction} показано взаимодействие среды исполнения с \name{Hy\-per\-led\-ger Iro\-ha} и локальной базой данных \name{PostgreSQL}, в которой хранится история транзакций, права доступа и информация о пользователях.
Среда исполнения принимает на вход код смарт-контракта и начинает его выполнение.
Для того, чтобы получать и записывать актуальное состояние переменных смарт-контракта, среда исполнения обращается к базе данных.
Запросы выполняются по необходимости.

\subsection{Реализация интерфейса взаимодействия}
\label{AddSmartContract}
Пользователи формируют транзакции с помощью специального набора команд.
Для работы с командами в коде используется паттерн <<фабричный метод>>.
Чтобы участники могли сохранять и выполнять код смарт-контракта, была добавлена новая команда \emph{Add\-Smart\-Con\-tract}.
Она включает в себя данные, необходимые для обращения к среде исполнения.
Содержание команды зависит от конкретной среды исполнения.
Для передачи транзакций и записи в блокчейн используются библиотеки \name{Protobuf} и \name{RapidJSON}\footnote{http://rapidjson.org/}, поэтому нужно уметь переводить данные, содержащиеся в AddSmartContract, как и в любой другой команде, в форматы \name{Protocol Buffers} и \name{JSON} и обратно во внутреннее представление.

Перед тем, как транзакция будет записана в истории блокчейна, все участники сети должны провести stateless и stateful валидацию всех команд внутри этой транзакции.
Для AddSmartContract stateless валидация может содержать различные проверки кода, например на синтаксическую корректность.
Во время stateful валидации участник сети обязан выполнить код и получить новое состояние блокчейна, чтобы в дальнейшем во время работы алгоритма консенсуса все участники договорились, принимать это новое состояние или нет.

\vspace{0.7cm}
\noindent{Таким образом, был создан программный интерфейс в виде новой команды Add\-Smart\-Con\-tract. 
Для обращения к среде исполнения смарт-контрактов нужно указать формат входных данных в реализации Add\-Smart\-Con\-tract, а также сделать запрос во время stateful валидации к среде исполнения и получить от неё результат выполнения кода смарт-контракта.}


\section{Внедрение среды исполнения смарт-кон\-трак\-тов}
В этой главе описаны аргументы для выбора среды исполнения и реализация взаимодействия этой среды с блокчейном \name{Hyperledger Iroha}.

\subsection{Выбор среды исполнения}
Сред исполнения смарт-контрактов огромное количество, необходимо выбрать одну из них.
Была выбрана виртуальная машина из проекта \name{Hyperledger Burrow}, так как она обладает следующими преимуществами: 1) выполнение байт-кода \name{EVM}; 2) наличие программного интерфейса для взаимодействия; 3) проект поддерживается \name{Hy\-per\-led\-ger}; 4) есть примеры интеграции с другими проектами \name{Hy\-per\-led\-ger Fabric}~\cite{HLFabricEVM} и \name{Hyperledger Sawtooth}~\cite{HLSeth}.
Так как \name{Ethereum} де-факто является самой популярной платформой для работы со смарт-контрактами, программистам будет проще адаптироваться к разработке на \name{Hy\-per\-led\-ger Iro\-ha}.
Важную роль играет принадлежность к \name{Hy\-per\-led\-ger} --- при возникновении проблем и вопросов на этапе интеграции виртуальной машины можно обратиться к сообществу разработчиков указанных ранее проектов.

Наличие программного интерфейса упрощает имплементацию схемы взаимодействия, описанной в подглаве~\ref{Interaction}.
У виртуальной машины внутри есть доступ к участку памяти memory, а для реализации операций над участком памяти storage необходимо обращаться к истории блокчейна.
Этот интерфейс описывает все методы, которые нужны виртуальной машине, чтобы получать, добавлять и обновлять данные блокчейна.


\subsection{Взаимодействие среды исполнения смарт-кон\-трак\-тов с \name{Hy\-per\-led\-ger Iro\-ha}}
В данной подглаве описано взаимодействие виртуальной машины \name{Hyperledger Burrow}, написанной на языке \name{Go}, с блокчейном \name{Hyperledger Iroha} в реализованной системе на примере жизненного цикла смарт-контракта.

Сначала пользователь добавляет в транзакцию команду Add\-Smart\-Con\-tract (см. подглаве~\ref{AddSmartContract}), передавая ей следующие данные: адрес вызывающего, адрес вызываемого, количество газа, код смарт-кон\-тракта и, если нужно вызвать уже существующую функцию, её аргументы.
После того, как пользователь завершил создание транзакции, она отправляется остальным участникам сети для проверки и последующего принятия или отклонения.

Далее транзакция должна пройти валидацию у каждого участника сети.
При stateful валидации команды Add\-Smart\-Con\-tract происходит вызов виртуальной машины \name{Hy\-per\-led\-ger Bur\-row} с заданными параметрами.
Для этого код на \name{C++} должен передавать данные в среду исполнения языка \name{Go} и получать из неё результат после соответсвующего запроса к виртуальной машине.
Связывание сред исполнения реализовано следующим образом.
На языке \name{Go} написана обёртка, внутри которой находится код, вызывающий виртуальную машину, и вспомогательные функции, реализующие интерфейс, необходимый для работы виртуальной машины.
Обёртка имеет доступ к базе данных \name{PostgreSQL}, в которой локально хранится история транзакций блокчейн сети.
Обёртка \achtung{собирается} компилятором \name{Go} с использованием опции \emph{buildmode}, которая создает разделяемую библиотеку (shared object) и заголовочный файл на языке \name{C} для работы с ней.

Таким образом, все участники независимо друг от друга запускают код и получают новое состояние блокчейна.
Далее, согласно алгоритму консенсуса, достигается соглашение о принятии транзакции в сеть, которое сообщается пользователю.

\section{Тестирование и апробация}
В этой главе описаны способы тестирования и апробации, которые использовались для проверки добавленной функциональности.

\subsection{Модульное тестирование}
В проекте \name{Hyperledger Iroha} есть фреймворк для тестирования на основе библиотеки \name{GTest}.
В этой подглаве описаны модульные тесты, которые были использованы для проверки корректности отдельных частей среды исполнения смарт-контрактов.

\subsection{Апробация}
Для проверки работоспособности системы в целом были реализованы следующие смарт-контракты: \name{AWallet}, \name{King Of The Ether Throne} и \name{Ponzi Scheme}. 
% https://ethereum.stackexchange.com/questions/2940/where-can-i-find-some-solidity-smart-contract-source-code-examples
\subsubsection{\name{AWallet}}
\subsubsection{\name{King Of The Ether Throne}}
\subsubsection{\name{Ponzi Scheme}}


\section{Результаты}
Была реализована инфраструктура поддержки среды исполнения смарт-контрактов для блокчейна \name{Hyperledger Iroha}.

В ходе работы были выполнены следующие задачи:
\begin{itemize}
    \item выполнен обзор существующих сред исполнения и языков смарт-контрактов, обзорная статья принята для публикации в Pro\-ceed\-ings of ISP RAS;
    \item разработаны архитектура и программный интерфейс для взаимодействия среды исполнения смарт-контрактов с \name{Hy\-per\-led\-ger Iro\-ha}
    \item реализовано взаимодействие \name{Hy\-per\-led\-ger Iroha} и среды исполнения смарт-контрактов из проекта \name{Hyperledger Burrow};
    \item проведено модульное тестирование и апробация на следующих смарт-контрактах: \name{AWallet}, \name{King Of The Ether Throne}, \name{Ponzi Scheme}.
        % http://kddlab.zjgsu.edu.cn:7200/research/blockchain/A%20Survey%20of%20Attacks%20on%20Ethereum%20Smart%20Contracts.pdf
        % https://ethereum.stackexchange.com/questions/2940/where-can-i-find-some-solidity-smart-contract-source-code-examples
\end{itemize}




\setmonofont[Mapping=tex-text]{CMU Typewriter Text}
\bibliographystyle{ugost2008ls}
\bibliography{diploma.bib}
\end{document}
