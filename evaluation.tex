\section{Тестирование}
В ходе работы в проект были добавлены два компонента: интерфейс для взаимодействия со средой исполнения смарт-контрактов (команда AddSmartContract) и виртуальная машина из проекта \name{Hy\-per\-led\-ger Bur\-row}.
В главе описано тестирование внедрённых команды AddSmartContract и реализации программного интерфейса, необходимого для корректной работы виртуальной машины.
В проекте \name{Hy\-per\-led\-ger Iro\-ha} существует инструмент тестирования \name{ITF} (Integration Test Fra\-me\-work), основанный на библиотеке \name{GTest}.
Он позволяет настроить блокчейн и его историю: сформировать аккаунты пользователей, выставить параметры для алгоритма консенсуса, создать заглушку, содержащую данные для формирования транзакций, провести stateless и stateful валидацию заранее заготовленной транзакции.

Для проверки команды Add\-Smart\-Con\-tract были добавлены модульные тесты, проверяющие операции над данными внутри команды, а именно: сериализация данных в форматы \name{Protocol Buffers} и \name{JSON} и десериализация из них --- а также интеграционные тесты на stateless и stateful валидацию.

Для тестирования реализации программного интерфейса, которая необходима виртуальной машине, были разработаны модульные тесты, содержащие вызов различных смарт-контрактов, в том числе и некорректных с точки зрения соответствия байт-коду \name{EVM}.
Внутри них выполняются базовые операции, такие как создание аккаунта-контракта, присваивание и чтение переменных, вызов функции с параметрами.
Критерием успеха являлась корректность состояния блокчейна после выполнения одного или нескольких смарт-контрактов в рамках одной транзакции.
