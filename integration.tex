\section{Внедрение среды исполнения смарт-кон\-трак\-тов}
В этой главе описаны аргументы для выбора среды исполнения и реализация взаимодействия этой среды с блокчейном \name{Hyperledger Iroha}.

\subsection{Выбор среды исполнения}
Сред исполнения смарт-контрактов огромное количество, необходимо выбрать одну из них.
Была выбрана виртуальная машина из проекта \name{Hyperledger Burrow}, так как она обладает следующими преимуществами: 1) выполнение байт-кода \name{EVM}; 2) наличие программного интерфейса для взаимодействия; 3) проект поддерживается \name{Hy\-per\-led\-ger}; 4) есть примеры интеграции с другими проектами \name{Hy\-per\-led\-ger Fabric}~\cite{HLFabricEVM} и \name{Hyperledger Sawtooth}~\cite{HLSeth}.
Так как \name{Ethereum} де-факто является самой популярной платформой для работы со смарт-контрактами, программистам будет проще адаптироваться к разработке на \name{Hy\-per\-led\-ger Iro\-ha}.
Важную роль играет принадлежность к \name{Hy\-per\-led\-ger} --- при возникновении проблем и вопросов на этапе интеграции виртуальной машины можно обратиться к сообществу разработчиков указанных ранее проектов.

Наличие программного интерфейса упрощает имплементацию схемы взаимодействия, описанной в подглаве~\ref{Interaction}.
У виртуальной машины внутри есть доступ к участку памяти memory, а для реализации операций над участком памяти storage необходимо обращаться к истории блокчейна.
Этот интерфейс описывает все методы, которые нужны виртуальной машине, чтобы получать, добавлять и обновлять данные блокчейна.


\subsection{Взаимодействие среды исполнения смарт-кон\-трак\-тов с \name{Hy\-per\-led\-ger Iro\-ha}}
В данной подглаве описано взаимодействие виртуальной машины \name{Hyperledger Burrow}, написанной на языке \name{Go}, с блокчейном \name{Hyperledger Iroha} в реализованной системе на примере жизненного цикла смарт-контракта.

Сначала пользователь добавляет в транзакцию команду Add\-Smart\-Con\-tract (см. подглаве~\ref{AddSmartContract}), передавая ей следующие данные: адрес вызывающего, адрес вызываемого, количество газа, код смарт-кон\-тракта и, если нужно вызвать уже существующую функцию, её аргументы.
После того, как пользователь завершил создание транзакции, она отправляется остальным участникам сети для проверки и последующего принятия или отклонения.

Далее транзакция должна пройти валидацию у каждого участника сети.
При stateful валидации команды Add\-Smart\-Con\-tract происходит вызов виртуальной машины \name{Hy\-per\-led\-ger Bur\-row} с заданными параметрами.
Для этого код на \name{C++} должен передавать данные в среду исполнения языка \name{Go} и получать из неё результат после соответсвующего запроса к виртуальной машине.
Связывание сред исполнения реализовано следующим образом.
На языке \name{Go} написана обёртка, внутри которой находится код, вызывающий виртуальную машину, и вспомогательные функции, реализующие интерфейс, необходимый для работы виртуальной машины.
Обёртка имеет доступ к базе данных \name{PostgreSQL}, в которой локально хранится история транзакций блокчейн сети.
Обёртка \achtung{собирается} компилятором \name{Go} с использованием опции \emph{buildmode}, которая создает разделяемую библиотеку (shared object) и заголовочный файл на языке \name{C} для работы с ней.

Таким образом, все участники независимо друг от друга запускают код и получают новое состояние блокчейна.
Далее, согласно алгоритму консенсуса, достигается соглашение о принятии транзакции в сеть, которое сообщается пользователю.
