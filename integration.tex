\section{Внедрение среды исполнения смарт-кон\-трак\-тов}
В этой главе описаны аргументы для выбора среды исполнения и реализация взаимодействия этой среды с блокчейном \name{Hyperledger Iroha}.

\subsection{Взаимодействие среды исполнения смарт-кон\-трак\-тов с \name{Hy\-per\-led\-ger Iro\-ha}}
В данном параграфе описано взаимодействие виртуальной машины \name{Hy\-per\-led\-ger Bur\-row}, написанной на языке \name{Go}, с блокчейном \name{Hy\-per\-led\-ger Iroha} в реализованной системе на примере жизненного цикла смарт-контракта.

Сначала пользователь добавляет в транзакцию команду Add\-Smart\-Con\-tract (см. па\-ра\-граф \ref{AddSmartContract}), передавая ей следующие данные: адрес вызывающего, адрес вызываемого, количество газа, код смарт-кон\-тракта и, если нужно вызвать уже существующую функцию --- её сигнатуру и аргументы.
Код передается в виде шестнадцатеричной последовательности, в которой закодированы  операции виртуальной машины.
Вызов функции кодируется следующим образом: на строковое представление сигнатуры функции применяется хэш \name{keccak256}, берутся первые четыре байта результата, а входные аргументы приписываются в конец.
После того, как пользователь завершил создание транзакции, она отправляется остальным участникам сети для проверки и последующего принятия или отклонения.

Далее транзакция должна пройти валидацию у каждого участника сети.
При stateful валидации команды Add\-Smart\-Con\-tract происходит обращение к виртуальной машине \name{Hy\-per\-led\-ger Bur\-row} с заданными параметрами.
Для этого код на \name{C++} должен передавать данные в среду исполнения языка \name{Go} и получать из неё результат после соответствующего запроса к виртуальной машине.
Связывание сред исполнения реализовано следующим образом.
На языке \name{Go} написана обёртка, внутри которой находится код, делающий запрос к  виртуальной машине, и вспомогательные функции, реализующие интерфейс, необходимый для работы виртуальной машины.
Обёртка имеет доступ к базе данных \name{PostgreSQL}, в которой локально хранится история транзакций блокчейн сети.
Обёртка транслируется компилятором \name{Go} с использованием опции \emph{buildmode}, которая создает разделяемую библиотеку (shared object) и заголовочный файл на языке \name{C} для работы с ней.

Таким образом, все участники независимо друг от друга запускают код и получают новое состояние блокчейна.
Далее, в соответствии с алгоритмом консенсуса, достигается соглашение о принятии или отклонении транзакции, которое сообщается пользователю.
