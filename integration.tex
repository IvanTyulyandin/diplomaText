\section{Внедрение среды исполнения смарт-кон\-трак\-тов}
В этой главе описаны аргументы для выбора среды исполнения и реализация взаимодействия этой среды с блокчейном \name{Hyperledger Iroha}.

\subsection{Выбор среды исполнения}
Сред исполнения смарт-контрактов огромное количество, необходимо выбрать одну из них.
Была выбрана виртуальная машина из проекта \name{Hyperledger Burrow}, так как она обладает следующими преимуществами: 1) выполнение байт-кода \name{EVM}; 2) наличие программного интерфейса для взаимодействия; 3) проект поддерживается \name{Hy\-per\-led\-ger}; 4) есть примеры интеграции с другими проектами \name{Hy\-per\-led\-ger Fabric}~\cite{HLFabricEVM} и \name{Hyperledger Sawtooth}~\cite{HLSeth}.
Так как \name{Ethereum} де-факто является самой популярной платформой для работы со смарт-контрактами, программистам будет проще адаптироваться к разработке на \name{Hy\-per\-led\-ger Iro\-ha}.
Наличие программного интерфейса упрощает имплементацию схемы взаимодействия, описанной в подглаве~\ref{Interaction}.
Важную роль играет принадлежность к \name{Hy\-per\-led\-ger} --- при возникновении проблем и вопросов на этапе интеграции виртуальной машины можно обратиться к сообществу разработчиков указанных ранее проектов.

\subsection{Взаимодействие среды исполнения смарт-кон\-трак\-тов с \name{Hy\-per\-led\-ger Iro\-ha}}
В данной подглаве описано взаимодействие виртуальной машины \name{Hyperledger Burrow} с \name{Hyperledger Iroha}.

