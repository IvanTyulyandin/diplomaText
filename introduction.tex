\section*{Введение}

В бионформатике необходимо обрабатывать большие объемы данных.
Одна из задач --- проверка гомологичности протеина семейству протеинов, 
то есть определение сходства.
Протеин состоит из произвольной по длине комбинации 20 стандартных аминокислот.
В биологии каждая аминокислота кодируется в виде символа.
Используя эту кодировку, можно получить символьную последовательность,
которая будет полностью описывать протеин.
Задача осложняется тем, что протеины, даже имея различную последовательность,
могут принадлежать к одному и тому же семейству.
Такое возможно из-за физико-биологических свойств стандартных аминокислот.

На основе известных данных о гомологичности группы протеинов 
можно построить \emph{профиль семейства} --- вероятностную модель, 
которая описывает семейство протеинов с помощью 
скрытой марковской модели (далее СММ) со специальной структурой,
учитывающей возможность вставки, замены и удаления аминокислот~\cite{HMM_Eddy}.
Для работы с СММ используется алгоритм Витерби, впервые предложенный в статье~\cite{Viterbi}.

С экспоненциальным ростом количества известных протеинов возрастает
необходимость ускорения их обработки.
Для решения задачи гомологичности используются различные варианты оптимизации,
начиная от вероятностных алгоритмов~\cite{MSV_Eddy} 
до использования графических процессоров общего назначения (GPGPU)~\cite{cudampf}.

В область зрения исследователей не попали методы, 
основанные на метавычислениях.
Иногда приходится работать с одним и тем же семейством протеинов.
Можно использовать этот факт для создания версии алгоритма Витерби, 
которая будет определять гомологичность только с конкретным семейством протеинов.
В теории метавычислений есть техника \emph{специализации}~\cite{Jones_spec}.
Параметры, подаваемые на вход алгоритму, можно условно разделить на две группы:
статические и динамические.
Статические параметры не меняются от запуска к запуску, 
а динамические могут принимать множество различных значений.
Суть специализации в подстановке известных статических 
данных в алгоритм и упрощении выражений, 
в которых все параметры известны после подстановки.
Ожидается, что специализированная программа будет работать быстрее за счет того, что часть вычислений, связанных с известными данными, будет выполнена на стадии генерации специализированной программы,
и их результат будет сохранен в коде этой программы.
Программа, генерирующая специализированную программу, называется \emph{специализатор}.

Для решения задачи гомологичности протеина с конкретным семейством протеинов 
можно сгенерировать специализированную версию алгоритма Витерби,
где СММ, описывающая семейство, будет статическим параметром, 
а протеин -- динамическим.
Возможно, что такая версия окажется оптимальнее по времени работы 
по сравнению с обобщенным алгоритмом Витерби.
