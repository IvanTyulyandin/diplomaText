\documentclass[hyperref={pdfpagelabels=false}]{beamer}
\hypersetup{unicode=true}
\usepackage{amssymb,cancel,cite,color,cmap,float,graphicx,lmodern,listings,
  multirow,pifont,pgfplots,txfonts,tikz,wrapfig,xcolor,yfonts}

\usepackage[T2A]{fontenc}
\usepackage[utf8]{inputenc}
\usepackage[english,russian]{babel}

% some weird help from tex compiler
\pgfplotsset{compat=1.15}

\usetikzlibrary{arrows,automata,positioning,shapes,shapes.multipart}
\usetheme{Copenhagen}

\setbeamertemplate{navigation symbols}{}
%\addtobeamertemplate{number}{}{%
%    \usebeamerfont{footline}%
%    \usebeamercolor[black]{footline}%
%    \hspace{1em}%
%    \insertframenumber%/\inserttotalframenumber
%}

%\expandafter\def\expandafter\insertshorttitle\expandafter{%
%  \insertshorttitle\hfill%
%  \insertframenumber}
\newcommand*\circled[1]{\tikz[baseline=(char.base)]{
    \node[shape=circle,draw,inner sep=2pt] (char) {#1};}}
\addtobeamertemplate{navigation symbols}{}{
  \usebeamerfont{footline}
  \fontsize{14pt}{14}\selectfont
  \usebeamercolor[black]{footline}
  \hspace{1em}
  \circled{${\insertframenumber}$}
}
%\textswab
\setbeamertemplate{frametitle}[default][center]
 
  
\title[Смарт-контракты для Hyperledger Iroha]{Проектирование и реализация среды исполнения смарт-контрактов для блокчейна Hyperledger Iroha}  
\author[И. Тюляндин]{Иван Тюляндин\\%
Научный руководитель: ст.\,преп. Я.\,А.\,Кириленко\\%
Консультант: к.\,ф.-м.\,н. Д.\,А.\,Березун\\%
Рецензент: ген.\,дир. “Сорамитсу Лабс” К.\,Р.\,Салахиев%
} 
\date{\today} 
\begin{document}
{
\setbeamertemplate{navigation symbols}{}
\frame[noframenumbering,plain]{
\maketitle
} 
}

\begin{frame}{Предметная область}
\begin{block}{Смарт-контракт} 
Программа, описывающая сделку участников блокчейн-сети. Запускается автоматически при достижении условий. 
Имеет преимущества перед бумажным контрактом.
\end{block}
\vfill
\begin{block}{Hyperledger Iroha}
Open-source блокчейн консорциума Hyperledger (https://github.com/hyperledger/iroha).
\end{block}
\end{frame} 

\begin{frame}{Цель}
Цель данной работы — создать прототип Hyperledger Iroha с возможностью выполнения смарт-контрактов.
\vfill
Смарт-контракты повысят доверие внутри сети и степень автоматизации договоров.
\end{frame} 

\begin{frame}{Поставленные задачи}
\begin{itemize}
\item Рассмотреть существующие языки и среды исполнения смарт-контрактов
\vfill
\item Разработать интерфейс для взаимодействия произвольной среды исполнения смарт-контрактов с Hyperledger Iroha
\vfill
\item Реализовать взаимодействие одной из сред исполнения с Hyperledger Iroha
\vfill
\item Провести тестирование
\end{itemize}

\end{frame} 

\begin{frame}{Интерфейс взаимодействия с Hyperledger Iroha}
Транзакция формируется с помощью команд управления.
\vfill
Нужна новая команда для работы со средой исполнения смарт-контрактов. 
Используется паттерн проектирования “Фабричный метод”.

\end{frame} 

\begin{frame}{Результаты}
\begin{itemize}
\item Выполнен обзор существующих сред исполнения и языков смарт-контрактов (статья на SYRCoSE)
\vfill
\item Разработан интерфейс для взаимодействия произвольной среды исполнения смарт-контрактов с Hyperledger Iroha
\vfill
\item Виртуальная машина Hyperledger Burrow внедрена
\vfill
\item Написаны модульные тесты 
\end{itemize}
\end{frame} 

\begin{frame}{Использованные технологии}
\begin{block}{Языки}
	C++, Go, Solidity
\end{block}
\vfill
\begin{block}{Библиотеки}
	Boost, Protobuf, RapidJSON, GTest, libpq
\end{block}
\vfill
\begin{block}{Дополнительно}
	PostgreSQL, EVM, CMake, Git, Docker
\end{block}

\end{frame} 

\end{document}
