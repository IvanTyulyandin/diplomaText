\section{Взаимодействие среды исполнения смарт-кон\-трак\-тов с \name{Hy\-per\-led\-ger Iro\-ha}}
В этой главе будет рассмотрена схема взаимодействия среды исполнения смарт-контрактов и программный интерфейс, который эту схему реализует.

\subsection{Схема взаимодействия}
\label{Interaction}

\begin{figure}[b!]
  \centering
  \includegraphics[width=0.6\columnwidth]{interaction.png}
  \caption{Диаграмма последовательности взаимодействия}
  \label{interaction}
\end{figure}

Для создания инфраструктуры для среды исполнения смарт-кон\-трак\-тов необходимо разработать схему взаимодействия этой среды и блокчейна \name{Hy\-per\-led\-ger Iro\-ha}.
Среда исполнения должна уметь получать и записывать данные в блокчейн.

На рисунке~\ref{interaction} показано взаимодействие среды исполнения с \name{Hy\-per\-led\-ger Iro\-ha} и локальной базой данных \name{PostgreSQL}, в которой хранится история транзакций, права доступа и информация о пользователях.
Среда исполнения принимает на вход код смарт-контракта и начинает его выполнение.
Для того, чтобы получать и записывать актуальное состояние переменных смарт-контракта, среда исполнения обращается к базе данных.
Запросы выполняются по необходимости.

\subsection{Реализация интерфейса взаимодействия}
\label{AddSmartContract}
Пользователи формируют транзакции с помощью специального набора команд.
Для работы с командами в коде используется паттерн <<фабричный метод>>.
Чтобы участники могли сохранять и выполнять код смарт-контракта, была добавлена новая команда \emph{Add\-Smart\-Con\-tract}.
Она включает в себя данные, необходимые для обращения к среде исполнения.
Содержание команды зависит от конкретной среды исполнения.
Для передачи транзакций и записи в блокчейн используются библиотеки \name{Protobuf} и \name{RapidJSON}\footnote{http://rapidjson.org/}, поэтому нужно уметь переводить данные, содержащиеся в AddSmartContract, как и в любой другой команде, в форматы \name{Protocol Buffers} и \name{JSON} и обратно во внутреннее представление.

Перед тем, как транзакция будет записана в истории блокчейна, все участники сети должны провести stateless и stateful валидацию всех команд внутри этой транзакции.
Для AddSmartContract stateless валидация может содержать различные проверки кода, например на синтаксическую корректность.
Во время stateful валидации участник сети обязан выполнить код и получить новое состояние блокчейна, чтобы в дальнейшем во время работы алгоритма консенсуса все участники договорились, принимать это новое состояние или нет.

\vspace{0.7cm}
\noindent{Таким образом, был создан программный интерфейс в виде новой команды Add\-Smart\-Con\-tract. 
Для обращения к среде исполнения смарт-контрактов нужно указать формат входных данных в реализации Add\-Smart\-Con\-tract, а также сделать запрос во время stateful валидации к среде исполнения и получить от неё результат выполнения кода смарт-контракта.}

