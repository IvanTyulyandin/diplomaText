\section{Текущие результаты}
В ходе работы был выполнен обзор предметной области.
Описаны терминология и технологии программирования гетерогенных систем.
Рассмотрена задача гомологичности.
Изучены проекты с открытым исходным кодом \name{HMMer} и \name{CUDAMPF}, 
которые решают задачу гомологичности с помощью скрытых марковских моделей и 
алгоритма Витерби.
Кратко изложена техника специализации GPGPU кода.

Реализовано чтение форматов fasta и hmm, используемых в \name{HMMer}, которые 
содержат информацию об исследуемых протеинах и скрытой марковской модели 
соответственно.
Запрограммирован однопоточный алгоритм Витерби для обработки фильтра MSV и 
дописывается многопоточная версия с использованием стандарта \name{SYCL} и 
компилятора \name{ComputeCpp}.
Исходный код выложен на \href{https://github.com/IvanTyulyandin/HMM_FASTA_Viterbi}{\name{GitHub}}.

Далее планируется перейти к написанию специализатора многопоточной версии 
алгоритма Витерби и сравнить производительность специализированной версии с 
неспециализированным вариантом на CPU, GPGPU, \name{HMMer} и \name{CUDAMPF}.
