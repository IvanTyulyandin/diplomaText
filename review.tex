\section{Обзор}
В этой главе будут рассмотрены среды исполнения и языки смарт-контрактов, а так же некоторые особенности \name{Hyperledger Iroha}.

\subsection{Языки и среды исполнения смарт-контрактов}
В 1997 году Ник Сабо (Nick Szabo) предложил концепцию смарт-контрактов~\cite{Szabo_SC}.
Смарт-контракт --- это программа, которая описывает взаимодействие участников блокчейн-сети. 
При сравнении с традиционным бумажным контрактом, смарт-контракт имеет однозначную семантику и выполняется автоматически при достижении определенных условий.
Результат выполнения смарт-контракта легко подтверждается, так как он будет записан в историю транзакций блокчейна.
На сегодняшний день существует множество различных языков смарт-контрактов и блокчейнов, которые могут исполнять программы на этих языках.

Смарт-контракт всегда должен завершаться для продолжения работы блокчейн-сети.
Если язык смарт-контрактов Тьюринг-полный, то среда исполнения должна предоставлять механизм, который будет ограничивать тем или иным способом количество операций смарт-контракта.
В случае \name{Ethereum} каждая инструкция стоит определенное количество \emph{газа}, цена которого выражена во внутренней криптовалюте \name{Et\-he\-re\-um}.
В \name{Hyperledger Fabric} имеется ограничение на время выполнения кода смарт-контракта.
Для языка \name{Rholang}~\cite{Rholang}, основанном на \name{Rho-calculus}~\cite{RhoCalculus}, выставляется лимит по количеству применений правил редукции.

Далее будут рассмотрены несколько сред исполнения смарт-кон\-трак\-тов и языки, которые данные среды поддерживают.

\subsubsection{Языки смарт-контрактов}
В этой подглаве рассмотрены языки смарт-контрактов с учетом их парадигм и свойств, таких как Тьюринг-полнота, механизм ограничения выполнения смарт-контракта платформой (на которой смарт-контракт исполняется) и системы типов.

На данный момент существуют языки смарт-контрактов с различным уровнем абстракции. 
\emph{Низкоуровневые языки} (low-level) предназначены для непосредственного выполнения средой исполнения.
Многие концепции, такие как семантика, вычислительная модель, система ограничения выполнения и типизация часто описываются на этом уровне.
Примеры таких языков --- \name{EVM}~\cite{EthereumYellowPaper}, \name{Bitcoin Script}~\cite{BitcoinScript} и \name{Michelson}~\cite{Michelson}.
\emph{Высокоуровневые языки} (high-level), такие как \name{Solidity}~\cite{Solidity}, \name{Flint}~\cite{Flint} и \name{Liquidity}~\cite{liquidity},
упрощают процесс разработки смарт-контрактов за счет повышенной читаемости, наличия более абстрактных синтаксических конструкций и системы типов.
\emph{Промежуточные языки} (intermediate-level) смарт-контрактов являются своего рода компромиссом между высокоуровневыми и низкоуровневыми языками по степени абстракции.
Как правило, они спроектированы для упрощения формальной верификации или статического анализа исходного кода (с учетом вычислительной модели, системы типов, семантики и других формализмов).
\name{Scilla}~\cite{Scilla} является промежуточным языком смарт-контрактов.

В ходе обзора языков смарт-контрактов на конференцию SYRCoSE 2019 была написана обзорная статья \emph{A Survey of Smart Contract Safety and Pro\-gram\-ming Languages}, которая будет опубликована в Proceedings of ISP RAS.
Ниже приведена сводная таблица по языкам смарт-кон\-трак\-тов и их свойствам из данной статьи.
В ней приведены следующие характеристики языков смарт-контрактов: название, уровень абстракции, текущее состояние разработки, проект (для которого язык предназначен), парадигма, способ ограничения выполнения смарт-контракта целевой платформой и Тьюринг-полнота соответственно. 


\begin{ThreePartTable}
\renewcommand\TPTminimum{\textwidth}
% Arrange for "longtable" to take up full width of text block
\setlength\LTleft{0pt}
\setlength\LTright{0pt}
\setlength\tabcolsep{0pt}
%\begin{TableNotes}
%\end{TableNotes}
\fontsize{11}{12}\selectfont
\begin{longtable}{  l @{\extracolsep{\fill}} *{6}{c} }
\toprule
    Language & Level & Current & Project
    & Paradigm /& Metering & Turing\\ 
    & & state & & influence & & completeness\\
\midrule
\endhead

\midrule[\heavyrulewidth]
\multicolumn{7}{r}{\textit{продолжение}}\\
\endfoot  

\midrule[\heavyrulewidth]
%\insertTableNotes  % tell LaTeX where to insert the table-related notes
\endlastfoot

Bamboo & high-level & alpha & Ethereum & functional 
& \centering {gas system} & yes\\
& & (experimental) & & & &\\
\addlinespace

Bitcoin & low-level & under & Bitcoin & stack-based, 
& script size & no\\
Script & & development & & reverse-polish & &\\
\addlinespace

Chain-& high-level & stable & Hyperledger & general purpose & timeout & yes\\
code & & & Fabric & languages & &\\
\addlinespace

EOSIO & high-level& stable & {EOS.IO} & object-oriented, & bound & yes\\
& & & & statically typed & system &\\    
\addlinespace

EVM & low-level& stable & Ethereum & stack-based & \centering gas system & yes\\
bytecode & & & & & &\\
\addlinespace

Flint & high-level & alpha & Ethereum & type safe, & {\centering gas system} & yes\\
& & & & contract-oriented& &\\
\addlinespace

IELE & low-level & prototype & Ethereum & register- & gas system & yes\\
& & & & based & &\\
\addlinespace

Ivy & high-level& prototype & Bitcoin & imperative & gas system & no\\
& & (experimental) & & & & \\
\addlinespace

Liquidity & high-level & under & Tezos & fully-typed, & gas system & yes\\
& & development & & functional & & \\
\addlinespace

LLL & intermediate-& under & Ethereum & stack-based & gas system & yes\\
& level & development & & & & \\
\addlinespace

Logikon & high-level & experimental & Ethereum & logical- & gas system & yes\\
& & & & functional & & \\
\addlinespace

Michelson & low-level & under & Tezos & stack-based, & gas system & yes\\
& & development & & strongly typed & & \\
\addlinespace

Plutus & high-level & under & Cardano & functional & gas system & yes\\
(PlutusCore) & (low-level) & development & & & & \\
\addlinespace

Rholang & high-level& under & RChain & functional & rule reduction & yes\\
& & development & & & system & \\
\addlinespace

Scilla & intermediate-& under & Zilliqa & functional & gas system & no\\
& level & development & & & & \\
\addlinespace

Simplicity & low-level & under & Bitcoin & functional, typed, & Bit Machine & no \\
& & development & & combinator-based & & \\
\addlinespace

Solidity & high-level & stable & Ethereum & statically typed, & gas system & yes\\
& & & & object-oriented & & \\
\addlinespace

SolidityX & high-level & beta & Ethereum & secure- & gas system & yes\\
& & & & oriented & & \\
\addlinespace

Vyper & high-level & beta & Ethereum & imperative & gas system & no\\
& & & & & & \\
\addlinespace

Yul & intermediate-& under & Ethereum & object-oriented & gas system & yes \\
& level & development & & & & \\
\addlinespace

\end{longtable}   
\end{ThreePartTable}

По результатам статьи было выявлено, что \name{Ethereum} является наиболее популярной платформой для работы со смарт-контрактами.
Экосистема данного блокчейна развита --- существует множество языков с различными подходами, сред разработки и статических анализаторов.
Аналогичной экосистемы нет ни у одного из всех рассмотренных блокчейнов.

\subsubsection{\name{Ethereum Virtual Machine}}
% https://habr.com/ru/post/340928/
\name{Ethereum Virtual Machine} (сокращенно \name{EVM}) --- виртуальная стековая машина блокчейна \name{Ethereum}~\cite{EthereumYellowPaper}.

\subsubsection{\name{Hyperledger Burrow}}
\name{Hyperledger Burrow} --- \name{EVM} с правами доступа, проект \name{Hy\-per\-led\-ger Fo\-un\-da\-ti\-on}.

\subsection{\name{Hyperledger Iroha}}
\subsubsection{Общие сведения о \name{Hyperledger Iroha}}
Транзакции формируются с помощью набора специальных команд.

\subsubsection{Валидация транзакции}
Прежде чем принять транзакцию, всем участникам сети нужно подтвердить её корректность.
Есть два этапа валидации --- \emph{stateless} и \emph{sta\-te\-ful}.
Транзакция должна быть сформирована согласно некоторым правилам.
Во время stateless валидации контролируется соответствие этим правилам.
Далее проводится stateful валидация. 
На этом этапе проверяется выполнимость отправленной транзакции (например, права доступа или наличие активов). 






