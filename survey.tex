\section{Обзор}
В разделе рассмотрены терминология и технологии программирования гетерогенных
вычислительных систем, в которых CPU управляет GPGPU, алгоритм Витерби и его
применение в задаче гомологичности.
Сделан обзор существующего программного обеспечения, решающего задачу
гомологичности.

\subsection[Технологии программирования гетерогенных систем] {Технологии
программирования гетерогенных\\вычислительных систем}
Здесь кратко описаны различные языки программирования систем вида 
CPU и GPGPU, а также их инфраструктура.

\subsubsection{\name{NVIDIA CUDA}}
\name{CUDA} (Compute Unified Device Architecture)~\cite{CUDA} --- 
это платформа параллельных вычислений и программный интерфейс 
для управления видеокартами.
\name{CUDA} является пропиетарной разработкой компании \name{NVIDIA}.
Исходный код на \name{CUDA} транслируется в \name{PTX} ---
псевдо-ассемблерное промежуточное представление, 
которое драйвер видеокарты переводит в бинарный код.
\name{CUDA} предназначена только для работы с GPGPU от \name{NVIDIA}.

\subsubsection{\name{OpenCL} и \name{SYCL}}
Разнообразие оборудования и интерфейсов усложняет поддержку
программного обеспечения.
Представителями индустрии была сформирована \name{Khronos Group} с целью
разработки общих открытых стандартов программирования гетерогенных
вычислительных систем.

Одним из результатов работы группы стал стандарт 
\name{OpenCL}~\cite{OpenCL} --- программный интерфейс для использования любого
устройства, которое предоставляет соответствующий драйвер.
Для разработки кода ядра используется диалект
языка \name{C99} с некоторыми ограничениями.
В \name{OpenCL} программист должен явно указывать, каким образом и когда
передавать данные на требуемое устройство.
Изначально ядро нужно было компилировать во время исполнения, 
то есть каждый раз при запуске программы, даже если само ядро не менялось.
Такой вариант называется онлайн компиляцией, которая необходима для 
обеспечения переносимости.

Позднее появился стандарт \name{SPIR-V}~\cite{SPIR-V}, описывающий
переносимое промежуточное представление.
\name{SPIR-V} можно транслировать в \name{LLVM IR}~\cite{LLVM} и наоборот.
Стало возможным единовременно компилировать весь исходный код с помощью 
инфраструктуры \name{LLVM}.
Способ, когда ядро во время сборки приложения компилируется в
\name{SPIR-V}, а затем используется \name{LLVM}, 
называется оффлайн компиляцией.

Наличие этих двух стандартов привело к созданию \name{SYCL}~\cite{SYCL}.
Это высокоуровневая абстракция на \name{С++} над \name{OpenCL} с полной
обратной совместимостью.
В отличие от \name{OpenCL}, в \name{SYCL} управление памятью автоматическое.
Программист должен указать, какие данные требуются для выполнения ядра на
конкретном устройстве.
Затем на стадии компиляции строится граф зависимости данных между 
ядрами, на основе которого генерируется код управления памятью.
Ядро пишется на стандартном \name{C++} с некоторыми ограничениями --- 
нельзя использовать исключения, указатели на функции и виртуальные функции, 
но можно применять лямбда-выражения, шаблоны, наследование и другие абстракции.
Такой дизайн (рис.~\ref{SYCL_infrastructure}) позволяет компилятору 
использовать систему типов \name{C++} для создания оптимизированного кода.
\begin{figure}
  \centering
  \includegraphics[width=\columnwidth]{sycl.jpg}
  \caption{Инфраструктура приложения с использованием \name{SYCL}~\cite{SYCL}}
  \label{SYCL_infrastructure}
\end{figure}

Сейчас есть четыре реализации стандарта \name{SYCL}: 
\name{ComputeCpp}~\cite{ComputeCpp} от Codeplay,
\name{DPC++}~\cite{DPC} от \name{Intel},
\name{hipSYCL}~\cite{hipSYCL},
\name{triSYCL}~\cite{triSYCL} от \name{AMD} и \name{Xilinx}.
Первая реализация распространяется бесплатно в виде разделяемой библиотеки,
остальные являются проектами с открытым исходным кодом.
На данный момент \name{ComputeCpp} является наиболее соответствующей 
стандарту реализацией.
Также идет работа над тем, чтобы транслировать код с использованием \name{SYCL} 
на эквивалентный \name{CUDA} код, так как приложения на \name{OpenCL} или
\name{SYCL} для GPGPU от \name{NVIDIA} менее производительны, чем аналогичные 
на \name{CUDA} из-за ограниченной поддержки стандарта \name{OpenCL} на этих 
устройствах со стороны \name{NVIDIA}.

\subsection{Задача гомологичности}
Еще раз про гомологичность, СММ.
\subsubsection{HMMer}
\subsubsection{Cudampf}

В теории, можно посмотреть в userguide HMMer.
