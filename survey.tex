\section{Обзор}
В разделе рассмотрены технологии программирования 
гетерогенных вычислительных систем, в которых CPU управляет GPGPU,
задача гомологичности и её существующие решения.

\subsection[Технологии программирования гетерогенных систем] {Технологии программирования гетерогенных\\вычислительных систем}
Здесь кратко описаны различные языки программирования систем вида 
CPU и GPGPU, а также их инфраструктура.

\subsubsection{\name{NVIDIA CUDA}}
\name{CUDA} (Compute Unified Device Architecture)~\cite{CUDA} --- 
это платформа параллельных вычислений и программный интерфейс 
для управления видеокартами.
\name{CUDA} является пропиетарной разработкой компании \name{NVIDIA}.
Исходный код на \name{CUDA} транслируется в \name{PTX} ---
псевдо-ассемблерное промежуточное представление, 
которое драйвер видеокарты переводит в бинарный код.
\name{CUDA} предназначена только для работы с GPGPU от \name{NVIDIA}.
%TODO как компилируется ядро? PTX появляется на стадии компиляции или в рантайме?
%TODO типобезопасность?

\subsubsection{\name{OpenCL} и \name{SYCL}}
Разнообразие оборудования и интерфейсов усложняет поддержку
программного обеспечения.
Представителями индустрии была сформирована \name{Khronos Group} с целью
разработки общих открытых стандартов программирования центральных процессоров 
и видеокарт.

Одним из результатов работы группы стал стандарт 
\name{OpenCL}~\cite{OpenCL} --- программный интерфейс для доступа к любому
устройству, которое предоставляет соответствующий драйвер.
Для разработки кода ядра используется диалект
языка \name{C99} с некоторыми ограничениями.
Изначально ядро нужно было компилировать во время исполнения, 
то есть каждый раз при запуске программы, даже если само ядро не менялось.
Такой вариант называется онлайн компиляцией, которая необходима для 
обеспечения переносимости.

Позднее появился стандарт \name{SPIR-V}~\cite{SPIR-V}, который описывает 
переносимое промежуточное представление.
\name{SPIR-V} можно перевести в \name{LLVM IR}~\cite{LLVM} и наоборот.
Стало возможным компилировать весь исходный код с помощью 
инфраструктуры \name{LLVM}.
Cпособ, когда ядро во время стадии компиляции исходного кода переводится в
\name{SPIR-V}, а затем используется \name{LLVM}, 
называется оффлайн компиляцией.


\subsection{Задача гомологичности}
Еще раз про гомологичность, СММ.
\subsubsection{HMMer}
\subsubsection{Cudampf}

В теории, можно посмотреть в userguide HMMer.
